\documentclass[11pt,a4paper]{article}

\usepackage[polish]{babel}
\usepackage[utf8]{inputenc}
\usepackage{polski}
\usepackage[T1]{fontenc}
\usepackage{indentfirst}
\usepackage{wrapfig}    % for wrapping figures, tables

\frenchspacing

%\usepackage{amsmath}
\usepackage{physics}
%\usepackage{bm}
\usepackage{gensymb}
%\usepackage{hepnames}
\usepackage{epsfig}
%\usepackage{amssymb}
\usepackage{graphics}
\usepackage[shortlabels]{enumitem}
%\usepackage{xspace}
%\xspaceaddexceptions{[]\{\}}

%
%
%fixpagesize
\pagestyle{empty}
\addtolength{\textwidth}{6cm}
\addtolength{\textheight}{4cm}
\addtolength{\evensidemargin}{-3cm}
\addtolength{\oddsidemargin}{-3cm}
\addtolength{\topmargin}{-2cm}
\parindent=0cm


%
%
%small distance in list/item/enum for enumitem package
\setlist[itemize,enumerate]{topsep=0em}
\setlist{noitemsep}

%print zadanie #
\newcounter{zadanie}\newcommand{\zadanie}[1][]{\addtocounter{zadanie}{1} ~\\  {\bf \emph{Zadanie \arabic{zadanie} #1 }} \\}
\newcounter{zaddom}\newcommand{\zaddom}[1][]{\addtocounter{zaddom}{1} ~\\  {\bf \emph{Zadanie domowe \arabic{zaddom} #1 }} \\}
%\renewcommand{\zadanie}[1][]{\pagebreak  ~\\  {\bf \emph{Zadanie }} \\} \addtolength{\topmargin}{-2cm}
\newcounter{wskazowka}\newcommand{\wskazowka}[1][]{\addtocounter{wskazowka}{1} ~\\  {\bf \emph{Wskazówka \arabic{wskazowka} #1 }} \\}
\newcounter{rozwiazanie}\newcommand{\rozwiazanie}[1][]{\addtocounter{rozwiazanie}{1} ~\\  {\bf \emph{Rozwiązanie: }} \\}
\newcommand{\dbar}{{\mkern3mu\mathchar'26\mkern-12mu d}}


%%%%%%%%%%%%%%%%%%%%%%%%%%%%%%%%%%%%%%%%%%%%%%%%%%%%%%
\begin{document}           % End of preamble and beginning of text.

\begin{centering}
\bf{\Large{Termodynamika z elementami fizyki statystycznej}}\\
Tydzień 6  (3 kwietnia 2024)\\[3mm]
energia wewnętrzna, ciepło, I zasada termodynamiki \\ 
\end{centering} 
\vspace{5mm}

%%%%%%%%%%%%%%%%%%%%%%%%%%%%%%%%%%%%%%%%%%%%%%%%%%%%%%%
%%%%%%%%%%%%%%%%%%%%%%%%%%%%%%%%%%%%%%%%%%%%%%%%%%%%%%%
\zadanie

Znajdź energię wewnętrzną $U(T)$ i równanie adiabaty gazu doskonałego o cieple molowym
$C_V = \frac{1}{n}\left( \frac{\rm{\dbar}Q}{\rm{d}T}\right)_V$.

\textbf{Wskazówki:}
\begin{itemize}
  \item korzystając z faktu, że
    $dU = \left(\frac{\partial U}{\partial T}\right)_V dT + \left(\frac{\partial U}{\partial V}\right)_T dV$
    pokaż, że $dU = n C_V dT$.
  \item w celu znalezienia równania adiabaty, skorzystaj z pierwszej zasady termodynamiki,
    oraz postaci $d(pV)$ dla gazu doskonałego\\
\end{itemize}


\vspace{5mm}
\rozwiazanie
Ciepło molowe przy stałej objętości definiujemy jako:
\begin{align}
C_V = \frac{1}{n}\left( \frac{\rm{\dbar}Q}{\rm{d}T}\right)_V,
\end{align}
gdzie ${\rm \dbar} Q$ oznacza infinitezymalny przyrost energii na sposób ciepła. \\
Pierwsza zasada termodynamiki:
\begin{align}
\rm{d}U = \rm{\dbar}Q + \rm{\dbar}W,
\end{align}
w warunkach izochorycznych przyjmuje postać:
\begin{align}
\rm{d}U_V = \rm{\dbar}Q _V, 
\end{align}
co implikuje,
\begin{align}
C_V = \frac{1}{n}\left( \frac{\rm{d}U}{\rm{d}T}\right)_V.
\end{align}
Z drugiej strony wiemy, że:
\begin{align*}
\rm{d}U = \left(\frac{\partial U}{\partial T} \right)_V \rm{d}T +  \left(\frac{\partial U}{\partial V} \right)_T \rm{d}V,
\end{align*}
dlatego dla $V=const.$ ciepło molowe $C_V$ jest równe:
\begin{align}
C_V = \frac{1}{n}\left( \frac{\partial U}{\partial T} \right)_V.
\end{align}
Po wycałkowaniu otrzymujemy:
\begin{align}
\Delta U = n C_V \Delta T 
\end{align}

Po wycałkowaniu od temperatury 0 otrzymujemy:
\begin{align}
U = n C_V T + U_{0}
\end{align}

Można się przekonać (np. stosując równanie energetyczne lub rozpatrując
rozprężanie gazu do próżni), że dla gazu doskonałego energia wewnętrzna nie
zależy od jego objętości. 

Równanie adiabaty $\rm{\dbar} Q =0$ znajdujemy korzystając ponownie z pierwszej zasady termodynamiki 
i wyrażenia na energię wewnętrzną uzyskanego powyżej:
\begin{align}
\rm{d}U = \rm{\dbar} W,
\end{align} 
a stąd
\begin{align}
n C_V \rm{d} T = - p \rm{d}V
\end{align}
Ponadto, z równania stanu gazu doskonałego (r. Clapeyrona) mamy:
\begin{align}
\rm{d}(pV) = n R \rm{d} T \implies  \rm{d} T = \frac{\rm{d}(pV)}{n R} = \frac{1}{n R}(p\rm{d}V + V\rm{d}p  )
\end{align}
i stąd
\begin{align}
- p \rm{d}V  = \frac{C_V}{R}\left(  p \rm{d}V +  V \rm{d}p \right)\\
- \left( 1 + \frac{R}{C_V}\right) p \rm{d}V = V \rm{d}p. \\
-\frac{C_P}{C_V}\frac{\rm{d} V}{V}=\frac{\rm{d} p}{p}
\end{align}
Ostatecznie:
\begin{align*}
-\frac{C_P}{C_V}\ln(V) = \ln(p) + const. \\
V^{-\frac{C_P}{C_V}} = const. \cdot p \implies pV^{\frac{C_P}{C_V}} = const.
\end{align*}
%%%%%%%%%%%%%%%%%%%%%%%%%%%%%%%%%%%%%%%%%%%%%%%%%%%%%
\newpage
%%%%%%%%%%%%%%%%%%%%%%%%%%%%%%%%%%%%%%%%%%%%%%%%%%%%%
\zadanie
W wielu omawianych dotąd procesach dla gazu doskonałego ciepło molowe jest stałe:
\begin{enumerate}
\item w przemianie izochorycznej ciepło molowe wynosi $C_V$,
\item w przemianie izobarycznej ciepło molowe wynosi $C_p = C_V + R$,
\item w przemianie adiabatycznej $\dbar Q = 0$, czyli ciepło molowe $C = 0$,
\item w przemianie izotermicznej $\rm{d}T = 0$, czyli ciepło molowe $C_T = \infty$.
\end{enumerate}
Wyprowadź wzór na ciepło molowe $C_X$ przy stałym $X$ dla gazu doskonałego.
$X$ jest dowolną funkcją zmiennych stanu. Następnie wyprowadź ogólne takiej 
przemiany (równanie politropy gazu doskonałego), zakładając $C_X=const$.

\textbf{Wskazówki:}
\begin{itemize}
\item skorzystaj z pierwszej zasady termodynamiki i różniczki zupełnej $\rm{d}U = n C_V \rm{d}T$.
\item znajdź wyrażenie na $n(C_X - C_V)$
\end{itemize}

\vspace{5mm}
\rozwiazanie
Ciepło molowe $C_X$ definiujemy jako:
\begin{align}
C_X = \frac{1}{n}\left( \frac{\rm{\dbar Q}}{\rm{d}T}\right)_X. \label{eq:cx}
\end{align}
Korzystając z pierwszej zasady termodynamiki, oraz postaci $\rm{d}U$ dla gazu doskonałego $\rm{\dbar} Q $ możemy zapisać jako:
\begin{align*}
\rm{\dbar } Q =  \rm{d} U -  \rm{\dbar} W =  n C_V \rm{d} T +  p \rm{d}V.
\end{align*}
Łącząc to z \eqref{eq:cx} otrzymujemy 
\begin{align}
C_X = C_V +\frac{p}{n} \left(\frac{\partial V}{\partial T} \right)_X \\
n(C_X - C_V) =  p \cdot \left(\frac{\partial V}{\partial T} \right)_X
\end{align}
Korzystając z równania stanu: 
\begin{align*}
\frac{n R T}{V} = p =n( C_X - C_V  ) \cdot \left(\left(\frac{\partial V}{\partial T} \right)_X \right)^{-1}
\end{align*}
Zatem:
\begin{align}
\left(\frac{\partial V}{V}\right)_X=\frac{C_X -C_V}{R} \left(\frac{\partial T}{T}\right)_X,
\end{align}
i po wycałkowaniu:
\begin{align}
\ln(V) = \frac{C_X -C_V}{R} \ln(T) + const.\\
V = C \cdot T^{\frac{C_X -C_V}{R}} \\
T V^{- \frac{R}{C_X-C_V}}=const.
\end{align}
Wykorzystując równanie stanu by wyrazić $T$ przez $p$, oraz związak $C_p = C_V + R$ otrzymujemy:
\begin{align}
  p V^{1- \frac{R}{C_X-C_V}}=
  p V^{\frac{C_X - C_V - R}{C_X-C_V}} = p V^{\frac{C_X - C_p}{C_X-C_V}} = p V^{\kappa }=const. \\
\end{align}
gdzie $\kappa$ oznacza wykładnik politropy:
\begin{align}
\kappa =\frac{C_X - C_p}{C_X - C_V}. \label{eq:kappa}
\end{align}
Przekształcając powyższą zależność otrzymujemy wzór na ciepło molowe rozważanej przemiany politropowej:
\begin{align}
\kappa (C_X - C_V) = C_X - C_p \implies C_X (\kappa - 1) = \kappa C_V = C_p \\
C_X = \frac{\kappa C_V -C_p}{\kappa -1}. \label{eq:cx2}
\end{align}

Przeanalizujmy przypadki szczególne:
\begin{itemize}
\item $\kappa = 0$  odpowiada przemianie izobarycznej:  $pV^{0} = const.$ stąd: $C_X = C_p$.
\item $\kappa =1$, odpowiada przemianie izotermicznej i   $pV^{1} = const.= nRT$  stąd: $C_X \rightarrow \infty$,
\item $\kappa = \frac{C_p}{C_V}$, odpowiada przemianie adiabatycznej i wtedy $C_X = 0$.
\item $\kappa \rightarrow \infty$: $pV^{\infty} = const. $ wtedy $C_X = C_V$, co opisuje przemianę izochoryczną.
\end{itemize}

\vspace{1cm}
Ponadto, przekształcając wzór \eqref{eq:cx2} do postaci:
\begin{align}
C_X =C_V \frac{\kappa - \frac{C_P}{C_V}}{\kappa-1},
\end{align}
można pokazać, że dla $\kappa \in  \left( 1,\frac{C_p}{C_V}\right)$ ciepło $C_X$ jest ujemne, co odpowiada za spadek temperatury gazu przy  jednoczesnym dostarczaniu do gazu ciepła. 

%%%%%%%%%%%%%%%%%%%%%%%%%%%%%%%%%%%%%%%%%%%%%%%%%%%%%
\newpage
%%%%%%%%%%%%%%%%%%%%%%%%%%%%%%%%%%%%%%%%%%%%%%%%%%%%%
\zadanie
Jeden mol gazu doskonałego przeszedł ze stanu opisanego parametrami
$p_0, V_0$ do stanu o objętości $V_1 = 2 V_0$.
Przemiana prowadzona była tak, że przez cały czas $p^2 V = const$.
Znajdź wykonaną nad układem pracę, zmianę energii wewnętrznej i ciepło wymienione z otoczeniem.
Jakie jest ciepło molowe w tej przemianie?
\vspace{5mm}
\rozwiazanie 

Z równania rozważanej przemiany mamy:
\begin{align}
p^2 V = p_0^2 V_0 \implies \quad p = p_0 \sqrt{\frac{V_0}{V}},
\end{align}
zatem praca wykonana nad układem jest równa:
\begin{align}
\Delta W = - \int_{V_0}^{2 V_0} p \rm{d} V =  - \int_{V_0}^{2 V_0}  p_0 \sqrt{\frac{V_0}{V}} \rm{d} V  = - 2 p_0 V_0 (\sqrt{2}-1).
\end{align}
Zmiana energii wewnętrznej gazu doskonałego jest równa:
\begin{align}
\Delta U =   C_V \Delta T = C_V \left( \frac{\sqrt{2}p_0 V_0}{R} -  \frac{p_0 V_0}{R} \right) = \frac{C_V}{R} p_0 V_0 ( \sqrt{2}-1).
\end{align}
Zatem ciepło wymienione w tym procesie jest równe:
\begin{align}
\Delta Q = \Delta U - \Delta W = p_0 V_0{(\sqrt{2}-1)\left[ \frac{C_V}{R} + 2 \right]} = \frac{p_0 V_0}{R}(\sqrt{2}-1) \left[ C_p + R\right]
\end{align}
Ciepło molowe tej przemiany:
\begin{align}
C_X = \frac{\Delta Q}{\Delta T } = \frac{ \frac{p_0 V_0}{R}(\sqrt{2}-1) \left[ C_V + 2 R\right]}{\frac{p_0 V_0}{R}(\sqrt{2}-1)}= C_V + 2 R = C_p + R.
\end{align}

Ciepło molowe można też wyznaczyć korzystając bazując na wynikach zadania 2. Teraz rozpatrujemy przemianę
\begin{align}
pV^{\frac{1}{2}}=const,
\end{align}
zatem mamy do czynienia z przemianą politropową o wykładniku politropy równym $$\kappa = \frac{1}{2}$$. Stąd:
\begin{align}
C_X = C_V\frac{\kappa - \frac{C_p}{C_V}}{\kappa-1}=C_V\frac{\frac{1}{2}-\frac{C_V+R}{C_V}}{-\frac{1}{2}}=-2C_V(\frac{1}{2}-1-\frac{R}{C_V})=2C_V(\frac{1}{2}+\frac{R}{C_V})=C_V+2R=C_p+R
\end{align}

%%%%%%%%%%%%%%%%%%%%%%%%%%%%%%%%%%%%%%%%%%%%%%%%%%%%%
\newpage
%%%%%%%%%%%%%%%%%%%%%%%%%%%%%%%%%%%%%%%%%%%%%%%%%%%%%
\zadanie
Oblicz energię wewnętrzną gazu van der Waalsa.
Założ, że molowe ciepło właściwe $C_V$ jest znane i nie zależy od temperatury.
Skorzystaj z następującej tożsamości (równanie energetyczne)
podanej bez dowodu {\em (dowód wymaga wprowadzenia pojęcia entropii)}:
\[ \left(\frac{\partial U}{\partial V}\right)_T = 
   T \cdot \left(\frac{\partial p}{\partial T}\right)_V - p.\]
\rozwiazanie 
Energia wewnętrzna zależy od dwu parametrów stanu, np. $U = U(T, V)$. Wtedy jej różniczka ma postać:
\begin{align}
  \rm{d} U &= \left(\frac{\partial U}{\partial T} \right)_V \rm{d}T  + \left(\frac{\partial U}{\partial V} \right)_T \rm{d}V \\
\end{align}

Pierwsza zasada termodynamiki:
\begin{align}
\rm{d}U = \rm{\dbar}Q + \rm{\dbar}W,
\end{align}
w warunkach izochorycznych przyjmuje postać:
\begin{align}
\rm{d}U_V = \rm{\dbar}Q _V, 
\end{align}
co implikuje,
\begin{align}
C_V = \frac{1}{n}\left( \frac{\rm{d}U}{\rm{d}T}\right)_V.
\end{align}

Różniczka $\rm{d}U$ ma więc ogólną postać:

\begin{align}
  \rm{d} U &= n C_V \rm{d}T + \left(\frac{\partial U}{\partial V} \right)_T \rm{d}V \\
 &= n C_V \rm{d}T + \left[    T \cdot \left(\frac{\partial p}{\partial T}\right)_V - p\right] \rm{d}V.
\end{align}
 Z drugiej strony, z równania van der Waalsa mamy:
 \begin{align}
 \left( \frac{\partial p}{\partial T} \right)_V = \frac{nR}{V - bn},
 \end{align}
 a stąd: 
 \begin{align}
  \rm{d} U &=  n C_V \rm{d} T + a \frac{n^2}{V^2} \rm{d}V.
 \end{align}
 Ostatecznie po wycałkowaniu:
 \begin{align}
 U(T, V) = n C_V T - a \frac{n^2}{V}  + U_0.
 \end{align}
%\wskazowka
 % Rozpisz wyrażenie na różniczkę zupełną $\rm{d}U$ korzystając z wyprowadzonego wcześniej zależności $n C_V = \left(\frac{\partial U}{\partial T} \right)_V$ oraz z przytoczonego powyżej równania energetycznego. 
 %\vspace{5mm}\\
%\textbf{Odpowiedź:} Energia wewnętrzna gazu van der Waalsa wyraża się %wzorem $ U(V, T) = n C_V T -\frac{n^2 a}{V} + U_0$.
  %\vspace{5mm}


%%%%%%%%%%%%%%%%%%%%%%%%%%%%%%%%%%%%%%%%%%%%%%%%%%%%%
\newpage
%%%%%%%%%%%%%%%%%%%%%%%%%%%%%%%%%%%%%%%%%%%%%%%%%%%%%
 
\zadanie
Korzystając z równania energetycznego:

\[ \left(\frac{\partial U}{\partial V}\right)_T = 
T \cdot \left(\frac{\partial p}{\partial T}\right)_V - p.\]

pokaż, że energia gazu doskonałego nie zależy od objętości, tj. $U=U(T)$.

\vspace{5mm}
\rozwiazanie

Energia wewnętrzna zależy od dwu parametrów stanu, np. $U = U(T, V)$. Wtedy jej różniczka ma postać:
\begin{align}
  \rm{d} U &= \left(\frac{\partial U}{\partial T} \right)_V \rm{d}T  + \left(\frac{\partial U}{\partial V} \right)_T \rm{d}V \\
\end{align}

Pierwsza zasada termodynamiki:
\begin{align}
\rm{d}U = \rm{\dbar}Q + \rm{\dbar}W,
\end{align}
w warunkach izochorycznych przyjmuje postać:
\begin{align}
\rm{d}U_V = \rm{\dbar}Q _V, 
\end{align}
co implikuje,
\begin{align}
C_V = \frac{1}{n}\left( \frac{\rm{d}U}{\rm{d}T}\right)_V.
\end{align}

Różniczka $\rm{d}U$ ma więc ogólną postać:

\begin{align}
  \rm{d} U &= n C_V \rm{d}T + \left(\frac{\partial U}{\partial V} \right)_T \rm{d}V \\
 &= n C_V \rm{d}T + \left[    T \cdot \left(\frac{\partial p}{\partial T}\right)_V - p\right] \rm{d}V.
\end{align}

 Natomiast z równania stanu gazu doskonałego: 
 \begin{align}
 \left(\frac{\partial p}{\partial T}\right)_V  = \frac{nR}{V} = p
 \end{align}
 
 co prowadzi do postaci $\rm{d} U$:
 \begin{align}
  \rm{d} U &= n C_V \rm{d}T + \left[T\frac{nR}{V} - p\right] \rm{d}V = \\
  &= n C_V \rm{d}T + \left[p - p\right] \rm{d}V = \\
  &= n C_V \rm{d}T
\end{align}  
 zatem energia wewnętrzna gazu doskonałego nie zależy od objętości.

 
%%%%%%%%%%%%%%%%%%%%%%%%%%%%%%%%%%%%%%%%%%%%%%%%%%%%%
\newpage
%%%%%%%%%%%%%%%%%%%%%%%%%%%%%%%%%%%%%%%%%%%%%%%%%%%%%
\zadanie
Gęstość energii wewnętrznej (energia wewnętrzna na jednostkę objętości: $u = U/V$) pewnego gazu jest 
wyłącznie funkcją temperatury: $u = u(T)$. Wiadomo też, że ciśnienie tego gazu to $p=\frac{1}{3}u$. 
Korzystając z równania energetycznego: 

\[ \left(\frac{\partial U}{\partial V}\right)_T = 
   T \cdot \left(\frac{\partial p}{\partial T}\right)_V - p.\]

Znajdź ogólną postać równania stanu i wyrażenie na energię wewnętrzną tego gazu.

\vspace{5mm}
\rozwiazanie
Wiemy, że
\begin{align}
u(T) = \frac{U}{V} \quad  \implies U = V \cdot u(T),
\end{align}
a stąd 
\begin{align}
\left( \frac{\partial U }{\partial V}\right)_T = u(T). \label{eq:part1}
\end{align}
Natomiast z równania energetycznego mamy,
\begin{align}
\left( \frac{\partial U }{\partial V}\right)_T  = T \cdot \left( \frac{\partial p }{\partial T}\right)_V - p = \frac{T}{3} \frac{\rm{d} u}{\rm{d} T}- \frac{u}{3}. \label{eq:part2}
\end{align}
Zatem 
\begin{align}
u = \frac{T}{3}\frac{\rm{d} u}{\rm{d} T}- \frac{u}{3}\\
\frac{\rm{d}u}{4 u} = \frac{\rm{d}T}{T}.
\end{align}
Po wycałkowaniu otrzymujemy:
\begin{align}
\frac{1}{4}\ln(u) = \ln(T) + const.\\
u = \alpha T^4,
\end{align}
gdzie $\alpha$ jest pewną stałą. 
Równanie stanu tej substancji dane jest wzorem:
\begin{align}
p = \frac{1}{3}u = \frac{\alpha}{3}T^4,
\end{align}
a energia wewnętrzna tej substancji:
\begin{align}
U = u V = \alpha V T^4
\end{align}
%%%%%%%%%%%%%%%%%%%%%%%%%%%%%%%%%%%%%%%%%%%%%%%%%%%%%
%%%%%%%%%%%%%%%%%%%%%%%%%%%%%%%%%%%%%%%%%%%%%%%%%%%%%
\end{document}
