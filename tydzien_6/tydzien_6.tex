\documentclass[11pt,a4paper]{article}

\usepackage[polish]{babel}
\usepackage[utf8]{inputenc}
\usepackage{polski}
\usepackage[T1]{fontenc}
\usepackage{indentfirst}
\usepackage{wrapfig}    % for wrapping figures, tables

\frenchspacing

%\usepackage{amsmath}
\usepackage{physics}
%\usepackage{bm}
\usepackage{gensymb}
%\usepackage{hepnames}
\usepackage{epsfig}
\usepackage{graphics}
\usepackage[shortlabels]{enumitem}
%\usepackage{xspace}
%\xspaceaddexceptions{[]\{\}}

%
%
%fixpagesize
\pagestyle{empty}
\addtolength{\textwidth}{6cm}
\addtolength{\textheight}{4cm}
\addtolength{\evensidemargin}{-3cm}
\addtolength{\oddsidemargin}{-3cm}
\addtolength{\topmargin}{-2cm}
\parindent=0cm


%
%
%small distance in list/item/enum for enumitem package
\setlist[itemize,enumerate]{topsep=0em}
\setlist{noitemsep}

%print zadanie #
\newcounter{zadanie}\newcommand{\zadanie}[1][]{\addtocounter{zadanie}{1} ~\\  {\bf \emph{Zadanie \arabic{zadanie} #1 }} \\}
\newcounter{zaddom}\newcommand{\zaddom}[1][]{\addtocounter{zaddom}{1} ~\\  {\bf \emph{Zadanie domowe \arabic{zaddom} #1 }} \\}
%\renewcommand{\zadanie}[1][]{\pagebreak  ~\\  {\bf \emph{Zadanie }} \\} \addtolength{\topmargin}{-2cm}

\newcommand{\dbar}{{\mkern3mu\mathchar'26\mkern-12mu d}}


%%%%%%%%%%%%%%%%%%%%%%%%%%%%%%%%%%%%%%%%%%%%%%%%%%%%%%
\begin{document}           % End of preamble and beginning of text.

\begin{centering}
\bf{\Large{Termodynamika z elementami fizyki statystycznej}}\\
Tydzień 6  (3 kwietnia 2023)\\[3mm]
energia wewnętrzna, ciepło, I zasada termodynamiki \\ 
\end{centering} 
\vspace{5mm}

%%%%%%%%%%%%%%%%%%%%%%%%%%%%%%%%%%%%%%%%%%%%%%%%%%%%
%%%%%%%%%%%%%%%%%%%%%%%%%%%%%%%%%%%%%%%%%%%%%%%%%%%%
\zadanie
Znajdź energię wewnętrzną $U(T)$ i równanie adiabaty gazu doskonałego o cieple molowym
$C_V = \frac{1}{n}\left( \frac{\rm{\dbar}Q}{\rm{d}T}\right)_V$.

\textbf{Wskazówki:}
\begin{itemize}
  \item korzystając z faktu, że
    $\rm{d}U = \left(\frac{\partial U}{\partial_T}\right)_V \rm{d}T + \left(\frac{\partial U}{\partial_V}\right)_T \rm{d}V$
    pokaż, że $\rm{d}U = n C_V \rm{d}T$.
  \item w celu znalezienia równania adiabaty, skorzystaj z pierwszej zasady termodynamiki,
    oraz postaci $\rm{d}(pV)$ dla gazu doskonałego\\
\end{itemize}
%%%%%%%%%%%%%%%%%%%%%%%%%%%%%%%%%%%%%%%%%%%%%%%%%%%%
%%%%%%%%%%%%%%%%%%%%%%%%%%%%%%%%%%%%%%%%%%%%%%%%%%%%
\zadanie
W wielu omawianych dotąd procesach dla gazu doskonałego ciepło molowe jest stałe:
\begin{enumerate}
\item w przemianie izochorycznej ciepło molowe wynosi $C_V$,
\item w przemianie izobarycznej ciepło molowe wynosi $C_p = C_V + R$,
\item w przemianie adiabatycznej $\dbar Q = 0$, czyli ciepło molowe $C = 0$,
\item w przemianie izotermicznej $\rm{d}T = 0$, czyli ciepło molowe $C_T = \infty$.
\end{enumerate}
Wyprowadź wzór na ciepło molowe $C_X$ przy stałym $X$ dla gazu doskonałego.
$X$ jest dowolną funkcją zmiennych stanu. Następnie wyprowadź ogólne równanie 
takiej przemiany (równanie politropy gazu doskonałego).

\textbf{Wskazówki:}
\begin{itemize}
\item skorzystaj z pierwszej zasady termodynamiki i różniczki zupełnej $\rm{d}U = n C_V \rm{d}T$.
\item znajdź wyrażenie na $n(C_X - C_V)$
\end{itemize}
%%%%%%%%%%%%%%%%%%%%%%%%%%%%%%%%%%%%%%%%%%%%%%%%%%%%
%%%%%%%%%%%%%%%%%%%%%%%%%%%%%%%%%%%%%%%%%%%%%%%%%%%%
\zadanie
Jeden mol gazu doskonałego przeszedł ze stanu opisanego parametrami
$p_0, V_0$ do stanu o objętości $V_1 = 2 V_0$.
Przemiana prowadzona była tak, że przez cały czas $p^2 V = const$.
Znajdź pracę wykonaną nad układem, zmianę energii wewnętrznej i ciepło wymienione z otoczeniem.
Jakie jest ciepło molowe w tej przemianie?

%%%%%%%%%%%%%%%%%%%%%%%%%%%%%%%%%%%%%%%%%%%%%%%%%%%%
%%%%%%%%%%%%%%%%%%%%%%%%%%%%%%%%%%%%%%%%%%%%%%%%%%%%
\zadanie
Oblicz energię wewnętrzną gazu van der Waalsa.
Założ, że molowe ciepło właściwe $C_V$ jest znane i nie zależy od temperatury.
Skorzystaj z następującej tożsamości (równanie energetyczne)
podanej na razie bez dowodu {\em (dowód wymaga wprowadzenia pojęcia entropii)}:
\[ \left(\frac{\partial U}{\partial V}\right)_T = 
   T \cdot \left(\frac{\partial p}{\partial T}\right)_V - p.\]

%%%%%%%%%%%%%%%%%%%%%%%%%%%%%%%%%%%%%%%%%%%%%%%%%%%%
%%%%%%%%%%%%%%%%%%%%%%%%%%%%%%%%%%%%%%%%%%%%%%%%%%%%   
\zadanie
Korzystając z równania energetycznego pokaż, że energia gazu doskonałego nie zależy od
objętości, tj. $U=U(T)$.

%%%%%%%%%%%%%%%%%%%%%%%%%%%%%%%%%%%%%%%%%%%%%%%%%%%%
%%%%%%%%%%%%%%%%%%%%%%%%%%%%%%%%%%%%%%%%%%%%%%%%%%%%
\zadanie
Gęstość energii wewnętrznej (energia wewnętrzna na jednostkę objętości: $u = U/V$) pewnego gazu jest 
wyłącznie funkcją temperatury: $u = u(T)$. Wiadomo też, że ciśnienie tego gazu to $p=\frac{1}{3}u$. 
Korzystając z równania energetycznego: 

\[ \left(\frac{\partial U}{\partial V}\right)_T = 
   T \cdot \left(\frac{\partial p}{\partial T}\right)_V - p.\]

Znajdź ogólną postać równania stanu i wyrażenie na energię wewnętrzną tego gazu.
%%%%%%%%%%%%%%%%%%%%%%%%%%%%%%%%%%%%%%%%%%%%%%%%%%%%
\newpage
%%%%%%%%%%%%%%%%%%%%%%%%%%%%%%%%%%%%%%%%%%%%%%%%%%%%

\begin{centering}
\bf{ Zadania domowe }\\[1mm]
\end{centering}
\vspace{1mm}

\zaddom
Zależność ciepła molowego $C_V$ wodoru od temperatury można przybliżyć w zakresie temperatur $0-300\,$K wzorem:
\[C_V(T) = \frac{3}{2}R + \frac{1}{2}R\cdot\left\{ 1 - \cos\left(\pi\cdot\frac{T}{300\,{\rm K}}\right)\right\}.\]
Jak zależy od temperatury energia wewnętrzna 1 mola wodoru w tym zakresie temperatur?

{\it Odpowiedź:} $U(T) = 2 RT - \frac{150 K R}{\pi} \sin\left(\frac{\pi T}{300K}\right) + U_{0}(V)$

\zaddom
Dla ciał stałych zależność ciepła molowego $C_V$ od temperatury można przybliżyć funkcją:
\[ C_V(T) = 3 R \cdot\frac{x^3}{(1+x^4)^{3/4}},\]
gdzie $x=T/T_0$, zaś $T_0$ jest temperaturą charakterystyczną dla danego ciała.
Jak zależy od temperatury energia wewnętrzna 1 mola ciała stałego?

{\it Odpowiedź:} $U(T) = 3R T_0 \left(\left(1+x^4\right)^{1/4}-1\right) + U_{0}(V)$

\zaddom
Wyprowadź wyrażenie na ściśliwość adiabatyczną ściśliwość, $\kappa_{ad}$ ,
kiedy gaz doskonały jest adiabatycznie oraz kwazistatycznie sprężany.
Następnie korzystając z wartości prędkości dźwięku wyznacz wykładnik adiabaty, 
$\kappa$ dla powietrza w warunkach normalnych:
pod ciśnieniem 1\,atm oraz o temperaturze 0\degree\,C.
Prędkość dźwięku jest dana równaniem $c=\sqrt{\dv{p}{\varrho}}$, gdzie $\varrho$ to gęstość.
Przyjmij $R = 8.21~$J/mol/K, prędkość dźwięku $v= 331.6$~m/s,
masa molowa suchego powietrza: $\mu = 28.96~$g/mol.
%wykładnik adiabaty suchego powietrza: $\kappa = 1.403$, 

{\it Odpowiedź:} Niech $pV^\kappa=const$ będzie równaniem adiabaty,wtedy
$\kappa_{ad} = \frac{1}{p \kappa}$, $\kappa = 1.403$.
%$c=\sqrt{\frac{R T \kappa}{\mu}} \approx332 \frac{m}{s}$ (powietrze w warunkach normalnych)

%%%%%%%%%%%%%%%%%%%%%%%%%%%%%%%%%%%
%%%%%%%%%%%%%%%%%%%%%%%%%%%%%%%%%%%
\zaddom
Gaz doskonały poddano przemianie,
w której spełniona była następująca zależność ciśnienia i temperatury:
\mbox{$p(T)=p_0 \exp(\frac{T}{T_0})$}, gdzie $p_0$ i $T_0$ są stałymi.
Wyznacz ciepło molowe oraz pracę wykonaną w tej przemianie, gdy gaz ogrzewany jest
od temperatury $T_1$ do $T_2$.

{\it Odpowiedź:}  $C_X = C_V + R\left(1 - \frac{T}{T_0}\right)$, $W = nR(T_1 - T_2 + \frac{1}{2}\frac{T_{2}^{2} - T_{1}^{2}}{T_{0}})$


\zaddom
Znajdź spadek temperatury $n$ moli gazu van der Waalsa przy adiabatycznym rozprężaniu do próżni od objętości $V_1$ do $V_2$.
Jaki byłby wynik tego typu rozprężania w przypadku gazu doskonałego?

{\it Odpowiedź:}
$T_1-T_2 = \frac{n a}{C_V} \left(\frac{1}{V_1}-\frac{1}{V_2}\right)$, 
dla gazu doskonałego $T_1-T_2 = 0$.

\end{document}
