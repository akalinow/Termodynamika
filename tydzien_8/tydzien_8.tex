\documentclass[11pt,a4paper]{article}

\usepackage[polish]{babel}
\usepackage[utf8]{inputenc}
\usepackage{polski}
\usepackage[T1]{fontenc}
\usepackage{indentfirst}
\usepackage{wrapfig}    % for wrapping figures, tables

\frenchspacing

%\usepackage{amsmath}
\usepackage{physics}
%\usepackage{bm}
\usepackage{gensymb}
%\usepackage{hepnames}
\usepackage{epsfig}
\usepackage{graphics}
\usepackage[shortlabels]{enumitem}
%\usepackage{xspace}
%\xspaceaddexceptions{[]\{\}}

%
%
%fixpagesize
\pagestyle{empty}
\addtolength{\textwidth}{6cm}
\addtolength{\textheight}{4cm}
\addtolength{\evensidemargin}{-3cm}
\addtolength{\oddsidemargin}{-3cm}
\addtolength{\topmargin}{-2cm}
\parindent=0cm


%
%
%small distance in list/item/enum for enumitem package
\setlist[itemize,enumerate]{topsep=0em}
\setlist{noitemsep}

%print zadanie #
\newcounter{zadanie}\newcommand{\zadanie}[1][]{\addtocounter{zadanie}{1} ~\\  {\bf \emph{Zadanie \arabic{zadanie} #1 }} \\}
\newcounter{zaddom}\newcommand{\zaddom}[1][]{\addtocounter{zaddom}{1} ~\\  {\bf \emph{Zadanie domowe \arabic{zaddom} #1 }} \\}
%\renewcommand{\zadanie}[1][]{\pagebreak  ~\\  {\bf \emph{Zadanie }} \\} \addtolength{\topmargin}{-2cm}

\newcommand{\dbar}{{\mkern3mu\mathchar'26\mkern-12mu d}}


%%%%%%%%%%%%%%%%%%%%%%%%%%%%%%%%%%%%%%%%%%%%%%%%%%%%%%
\begin{document}           % End of preamble and beginning of text.

\begin{centering}
\bf{\Large{Termodynamika z elementami fizyki statystycznej}}\\
Tydzień 8 (20 kwietnia 2023)\\[3mm]
Entropia jako funkcja stanu, podstawowe r. termodynamiki, współrzędne $T-S$\\
\end{centering} 
\vspace{5mm}

\zadanie
Oblicz entropię $n$ moli gazu doskonałego. Korzystając z wyprowadzonego
wzoru znajdź równanie adiabaty we współrzędnych $p - V$.
Znajdź równanie izochory  i izobary we współrzędnych $T-S$.

\zadanie
Jeden mol gazu doskonałego rozpręża się odwracalnie od objętości początkowej $V_1$ do objętości końcowej $V_2$ w ten sposób, że $p = \alpha V$, gdzie $\alpha=const$. Znaleźć zależność $T(S)$ oraz ciepło $Q$ dostarczone w trakcie tego procesu.

\zadanie
Znajdź różnicę entropii jednego grama azotu gazowego $N_2$ 
w temperaturze 20\degree C i ciekłego w -196\degree C, pod tym samym ciśnieniem 1 atm. 
Ciepło parowania $c$=200\,kJ/kg. Traktuj azot w stanie lotnym jako gaz doskonały, 
o stałym ciepłe molowym przy stałym ciśnieniu $C_P$=29,3 J/mol/K i o masie molowej $\mu=28$g/mol.           


%\zadanie
%Energia swobodna $F = U - T S$ ~jednego mola pewnego gazu wyraża się wzorem:
%\[ F(T,V) = \frac{3}{2} R T (1 - \ln T) - R T \ln (V-b) - \frac{a}{V},\]
%gdzie $a$ i $b$ są pewnymi stałymi.
%Znaleźć entropię, energię wewnętrzną i równanie stanu tego gazu.

\zadanie
Znając entropię pewnej (akademickiej) substancji $pV$: $S(U,V,N) = 3A(UVN)^\frac{1}{3}$ 
znajdź jej równanie stanu \mbox{$f(T,V,p)=0$.}

\zadanie
Wyprowadzić tożsamość, z której korzystaliśmy przy okazji znajdowania energii wewnętrznej gazu van der Waalsa:
\[ \left(\frac{\partial U}{\partial V}\right)_T = 
   T \cdot \left(\frac{\partial p}{\partial T}\right)_V - p \]
{\em Wskazówka: Traktując energię wewnętrzną $U$ oraz entropię $S$ 
jako funkcje parametrów stanu $V$ i $T$, zapisać ich różniczki zupełne,
skorzystać z podstawowego r. termodynamiki
% i skorzystać z I zasady termodynamiki: 
%$dU = \dbar Q + \dbar W = T dS - p dV$ (tożsamość termodynamiczna).
i z równości pochodnych mieszanych $\frac{\partial^2 S}{\partial V\partial T}$}.

\zadanie
Znajdź wyrażenie na entropię gazu fotonowego. 
Z jego pomocą wyznacz równanie adiabaty dla tego gazu.


\zadanie
Równanie stanu paramagnetyka (np. w polu solenoidu) zapisać można jako:
$m= C \frac{H}{T}$, 
gdzie $C$ - stała Curie, $H$ - natężenia pola magnetycznego, $T$- temperatura,
$m$ - magnetyzacja, tj. moment magnetyczny $M$ w jednostce objętości ($m=M/V$).
Przy wykluczeniu energii oddziaływania paramegnetyk-solenoid, praca zapisana może być jako
 $\dbar W = -\mu_0 M dH$, zaś energia wewnętrzna: $U=-\mu_0 M H$. Znajdź entropię paramagnetyka.

%\zadanie
%Wyprowadzić tożsamość, z której korzystaliśmy przy okazji znajdowania energii wewnętrznej gazu van der Waalsa:
%\[ \left(\frac{\partial U}{\partial V}\right)_T = 
%   T \cdot \left(\frac{\partial p}{\partial T}\right)_V - p \]
%{\em Wskazówka: Traktując energię wewnętrzną $U$ oraz entropię $S$ jako funkcje parametrów stanu $V$ i $T$, zapisać ich różniczki zupełne i skorzystać z I zasady termodynamiki: $dU = \dbar Q + \dbar W = T dS - p dV$. Następnie, w celu pokazania
%że $\left(\frac{\partial S}{\partial V}\right)_T = \left(\frac{\partial p}{\partial T}\right)_V$, wprowadzić funkcję stanu $F = U - T S$ i policzyć jej różniczkę
%zupełną oraz drugą pochodną mieszaną $\frac{\partial^2 F}{\partial V\partial T}$}.

\pagebreak
\zaddom
Oblicz entropię $n$ moli gazu van der Waalsa. Korzystając z wyprowadzonego
wzoru znajdź równanie adiabaty we współrzędnych $p - V$.

\vspace{5mm}
{\em Odpowiedź:} Entropia $S = n C_V \ln T + nR \ln (V - nb) + S_0$, r. adiabaty: $(p + n^2 a/ V)(V-nb)^{C_p/C_V} = {\rm const}$.


\zaddom
Jeden mol gazu van der Waalsa o temperaturze początkowej $T_1$ rozpręża się
adiabatycznie, w sposób odwracalny, od objętości $V_1 = 101\cdot b$ do
$V_2 = 1001\cdot b$. Oblicz temperaturę końcową gazu. Przyjmij, że
molowe ciepło właściwe jest stałe i równe $C_V=\frac{5}{2}R$.\\[3mm]
{\em Wskazówka: entropia gazu van der Waalsa dana jest wyrażeniem:\\
$\displaystyle S(T,V) = n C_V \ln{T} + n R \ln{(V-nb)} + S_0.$ }

\vspace{5mm}
{\em Odpowiedź:} Temperatura końcowa wynosi: $T_2 = 10^{-2/5} T_1$.  

\zaddom
Znaleźć zmianę entropii $S$ jednego mola gazu doskonałego oraz gazu van der Waalsa w procesie 
izotermicznym, jeżeli ciśnienie początkowe wyniosło $p_1$, a końcowe $p_2$. 
Założyć, że w obu przypadkach $C_V$ jest znaną stałą.
Ponadto dla gazu van der Waalsa założyć, że znane jest $V_1$, $V_2$ oraz stała $b$.
Zadanie rozwiąż, korzystając z wprost z definicji zmiany entropii 
w infinitezymalnym procesie odwracalnym. 

\vspace{5mm}
{\em Odpowiedź:} Zmiana entropii wynosi $\Delta S = R \ln[ (V_2 - b)/(V_1 - b)]$. 

\zaddom
%{\bf\em (Skrypt, str.105) }\\
Entropia $n$ moli pewnego gazu wynosi:
\[ S(U,V) = \frac{n}{2}\left( 5 R \ln{\frac{U}{n}} + 2 R \ln{\frac{V}{n}} + S_0 \right)\!,\]
gdzie $S_0=const$. Udowodnij, że gaz ten spełnia równanie stanu gazu doskonałego
i znajdź jego molowe ciepło właściwe $C_V$.

\vspace{5mm}
{\em Odpowiedź:} Ciepło molowe $C_V = 5/2 R$. 


\zaddom
Dane jest równanie stanu pewnej nowej substacji: $p=A T^3 /V$, gdzie $A$ - stała. 
Znana jest też energia wewnętrzna tej substancji: $U= B T^n \ln(V/V_0)+ f(T)$,
gdzie $V_0, B, n$ - stałe, zaś $ f(T)$ - zależy tylko od $T$. Znajdź $B$ i $n$.

\vspace{5mm}
{\em Odpowiedź:} $B= 2A/V_0$, $n=3$.

\end{document}

