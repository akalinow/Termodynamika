\documentclass[11pt,a4paper]{article}

\usepackage{polski}
%kodowanie widnows
%\usepackage[cp1250]{inputenc} 
%kodowanie linux
%\usepackage[utf8}{inputenc}


%\usepackage[T1]{fontenc}
\usepackage{indentfirst}
\usepackage{wrapfig}    % for wrapping figures, tables

\frenchspacing

\usepackage{amssymb}
%\usepackage{bm}
\usepackage{gensymb}
%\usepackage{hepnames}
\usepackage{epsfig}
\usepackage{graphics}
\usepackage[shortlabels]{enumitem}
%\usepackage{xspace}
%\xspaceaddexceptions{[]\{\}}

%
%
%fixpagesize
\pagestyle{empty}
\addtolength{\textwidth}{6cm}
\addtolength{\textheight}{5cm}
\addtolength{\evensidemargin}{-3cm}
\addtolength{\oddsidemargin}{-3cm}
\addtolength{\topmargin}{-2cm}
\parindent=0cm


%
%
%small distance in list/item/enum for enumitem package
\setlist[itemize,enumerate]{topsep=0em}
\setlist{noitemsep}

%print zadanie #
\newcounter{zadanie}\newcommand{\zadanie}[1][]{\addtocounter{zadanie}{1} ~\\  {\bf \emph{Zadanie \arabic{zadanie} #1 }} \\}
\newcounter{zaddom}\newcommand{\zaddom}[1][]{\addtocounter{zaddom}{1} ~\\  {\bf \emph{Zadanie domowe \arabic{zaddom} #1 }} \\}
%\renewcommand{\zadanie}[1][]{\pagebreak  ~\\  {\bf \emph{Zadanie }} \\} \addtolength{\topmargin}{-2cm}


%%%%%%%%%%%%%%%%%%%%%%%%%%%%%%%%%%%%%%%%%%%%%%%%%%%%%%
\begin{document}           % End of preamble and beginning of text.
\vspace*{-1.8cm}

\begin{centering}
\bf{\Large{Termodynamika z elementami fizyki statystycznej}}\\
Ćwiczenia 2 (7 marca 2024)\\[1mm]
własności cieplne cd. \\
\end{centering}

\zadanie
Przy długości fali $\lambda = 0.7\,\mu$m porównano natężenie promieniowania dwóch,
doskonale czarnych, źródeł promieniowania o różnych temperaturach.
Temperatura pierwszego ciała wynosi $T_1 = 1068 \degree$C (topnienie złota). 
Znaleźć temperaturę drugiego ciała $T_2$, jeśli stosunek natężeń promieniowania 
wynosił $I_\lambda (T_2)/I_\lambda (T_1) = 10$.
Przyjmij $hc = 1.24$~eV$\mu$m, $k_{B} = 8.6\cdot 10^{-5}$~eV/K

\zadanie [(Pirometr dwubarwny)]
Wyznaczyć temperaturę ciała świecącego wiedząc, że stosunek natężeń promieniowania dla długości fal
$\lambda_1=550$\,nm i $\lambda_2=700$\,nm wynosi $R=I_{\lambda_2}/I_{\lambda_1}(T)=1.286$.
Przyjąć $T \sim 10^{3}$~K.

\zadanie

Wyznaczyć temperaturę Ziemi zakładając, że znajduje się ona w równowadze radiacyjnej ze
Słońcem. Temperatura Słońca wynosi $T_\odot = 5800$~K, a promień $R_\odot = 7 \cdot 10^8$~m.
Odległość między Ziemią a Słońcem wynosi $D = 1.5 \cdot 10^{11}$~m. 
Przyjąć, że temperatura na powierzchni Ziemi jest stała podczas całego cyklu dobowego.
Obliczenia należy wykonać w dwóch wariantach:\\
\begin{enumerate}[a)]
        \item Ziemia jest ciałem doskonale czarnym i nie posiada atmosfery.
        \item Ziemia jest ciałem doskonale czarnym i posiada atmosferę w postaci sfery o promieniu niewiele większym od promienia Ziemi. 
        W modelu tym $\alpha=40\%$ promieniowania pochodzącego ze Słońca odbija się od górnych warstw atmosfery, zaś pozostałe $60\%$ 
        przenika przez atmosferę i zostaje pochłonięte przez Ziemię. Atmosfera pochłania całe promieniowanie wyemitowane przez Ziemię
  (promieniowanie Ziemi ma inny zakres długości fal niż promieniowanie Słońca). Atmosfera jest w równowadze tylko z Ziemią, 
  ponieważ nie absorbuje promieniowania słonecznego.
\end{enumerate}
%Otrzymany wynik należy porównać z rzeczywistą średnią temperaturą
%Ziemi równą $15^\circ C$.
{\it Wskazówka: }
W wariancie b) Ziemia znajduje się w równowadze radiacyjnej ze Słońcem i atmosferą. 
Atmosfera natomiast jest w równowadze tylko z Ziemią, ponieważ nie absorbuje promieniowania słonecznego.


\zadanie [(Termos próżniowy I)]
Dane są dwie nieskończone doskonale czarne płaszczyzny o temperaturach $T_1=300\,$K i $T_2=4$\,K.
Obliczyć strumień energii (czyli moc na jednostkę powierzchni)
przesyłaną między nimi. Rozważyć trzecią płaszczyznę (osłonę)
między nimi, która odbija $R$ = 95\% promieniowania. 
Obliczyć temperaturę osłony i strumień energii pomiędzy płaszczyznami.
Przyjmij $\sigma = 5.67\cdot 10^{-8}~\mathrm{W/m^{2}K^{4}}$

\zadanie [(Termos próżniowy II)]
Dwie równoległe, duże, doskonale czarne płyty umieszczone są w próżni i mają temperatury $T_1$ i $T_2$.
Między te płyty wstawiamy równolegle do nich $n$ dużych, cienkich, doskonale czarnych płyt.
Jaka jest temperatura $i$-tej płyty? 
Ile razy, w wyniku wstawienia płytek, zmniejszy się strumień energii pomiędzy płaszczyznami? 

%\textbf{Odpowiedź:}
%Temperatura $i$-tej płyty $T_i^{4} = T_1^4 + \frac{T_2^4-T_1^4}{n+1} (i-1)$.
%Zmiana strumienia: $I_n/I_2 = \frac{1}{n+1} $

\zadanie
Korzystając z prawa promieniowania Plancka wykaż, bez całkowania, prawo Stefana-Boltzmanna.

\pagebreak
\begin{centering}
\bf{ Zadania domowe }\\[1mm]
\end{centering}
\vspace{1mm}

\zaddom
Zależność ciśnienia równowagi fazy ciekłej i lotnej opisuje 
w przybliżeniu wzór: $ p \,=\, A e^{-\alpha /T} $.\\
Dla wody: $p_3 = 612\,$Pa, $T_3 = 273.16\,$K, 
$p_{\rm wrzenia} = 1.013\cdot 10^5\,$Pa, 
$T_{\rm wrzenia} = 373.2\,$K.\\ 
Wyznacz stałe $A$ i $\alpha$.
Oblicz w jakiej temperaturze woda wrze na wysokościach:
\begin{enumerate}[a)] 
\item 2500\,m -- Rysy ($p = 0.75\,$bar)
\item 4800\,m -- Mont Blanc ($p = 0.55\,$bar)
\item 8850\,m -- Mont Everest ($p = 0.33\,$bar).
\end{enumerate}
Przy jakim ciśnieniu woda wrze w temperaturze $20^\circ$C\,? $-3^\circ$C\,?
\vskip 10pt
\textbf{Odpowiedź:}
 $\alpha = \log\left(\frac{p_w}{p_3}\right)\frac{T_wT_3}{T_w-T_3} \approx 5208 K$, $A= p_w e^{\alpha/T_w} \approx 1.17 \cdot 10^{11} Pa$\\
$T_a \approx 367 K$, $T_b \approx 358 K$, $T_c \approx 346 K$\\
$P(20^\circ C) \approx 2243 Pa$, $P(-3^\circ C) \approx 494 Pa$

\zaddom
Do budowy termoregulatorów, ograniczników temperatury i tym
podobnych urządzeń stosuje się często urządzenie zwane bimetalem.
Jest to pasek złożony z dwóch spojonych ze sobą warstw metali
o różnych współczynnikach rozszerzalności.
Pasek taki przy ogrzewaniu będzie się wyginał i może
w ten sposób zamykać lub otwierać obwód elektryczny.
Dany jest bimetal o grubości $d$, złożony z metali
o współczynnikach rozszerzalności liniowej $\alpha_1$ i $\alpha_2$
($\alpha_1 > \alpha_2$).
W temperaturze $T_0$ bimetal jest prosty.
Znajdź promień krzywizny bimetalu po ogrzaniu go o $\Delta T$.
Wykonaj obliczenia dla: $\alpha_1=1.8\cdot 10^{-5}\,{\rm K}^{-1}$ (mosiądz),
$\alpha_2=1.2\cdot 10^{-5}\,{\rm K}^{-1}$ (stal), grubość $d=1\,$mm,
długość $l_0=5\,$cm, $\Delta T=100\,$K.
\vskip 10pt
\textbf{Odpowiedź:}
$R = \frac{d}{2} \frac{2+(\alpha_1+\alpha_2)\Delta T}{(\alpha_1-\alpha_2)\Delta T} \approx 167 cm$

\zaddom
Opór właściwy miedzi zależy od temperatury jak:
$ \rho(T) = 
  A \left(\frac{T}{T_0}\right) \tanh^3 \left(\frac{T}{T_0}\right);
  ~~ T_0 = 87\,\rm{K}. 
$
W temperaturze 290\,K miedziany czujnik ma opór 10\,$\Omega$. 
\begin{enumerate}
\item Jaki opór ma ten czujnik w temperaturze 700\,K? 
\item Jak zmieni się opór dla temperatury 701\,K? Ile wynosi $\Delta R/R$? 
\item Jaki jest opór w temperaturze 20\,K?
\item Jak zmieni się opór dla temperatury 21\,K? Ile wynosi  $\Delta R/R$?
\end{enumerate}
Zadanie rozwiąż rachunkiem bezpośrednim oraz korzystając z odpowiednich rozwinięć.

\vskip 10pt
\textbf{Odpowiedź:}
\begin{enumerate}
\item $\rho(700 K) = 24.323 \Omega$
\item Bezpośrednio $\rho(701 K) = 24.358 \Omega$\\
 Przy użyciu rozwinięcia $\rho(701 K) \approx \rho(700 K) + \rho'(700 K) 1 K = 24.358 \Omega$\\
 $\Delta R/R \approx 0.0014$
\item $\rho(20 K) = 0.008 \Omega$
\item Bezpośrednio $\rho(21 K) = 0.0097 \Omega$\\
 Przy użyciu rozwinięcia $\rho(21 K) \approx \rho(20 K) + \rho'(20 K) 1 K = 0.00957 \Omega$\\
 $\Delta R/R \approx 0.21$
\end{enumerate}

%%%%%%%%%%%%%%%%%%%%%%%%%
\newpage
%%%%%%%%%%%%%%%%%%%%%%%%%

\zaddom
Oszacuj całkowitą moc jaką wypromieniowujesz. Opisz przyjęte założenia i zastosowane przybliżenia. Oszacuj
wydatek energetyczny organizmu na utrzymanie temperatury ciała (różnicę pomiędzy mocą wypromieniowywaną
i otrzymywaną) jeżeli znajdujesz się w otoczeniu o temperaturze $20\degree$C.
\vskip 10pt
\textbf{Odpowiedź:}
$\Delta P = S \sigma (T_c^4-T_o^4) \approx 211 W$, $\lambda_{max} = 9.5 \mu m$

\zaddom
Sonda kosmiczna o kształcie kuli i doskonale czarnej powierzchni ma zbadać okolice Merkurego. Aby uniknąć
przegrzania sondy wyposażono ją w ekran termiczny - cienką osłonkę o kształcie półsfery zrobioną z metalu
o współczynniku odbicia $r$. Osłona założona jest bardzo blisko powierzchni sondy, ale nie styka się z nią. Sonda
zwrócona jest osłoniętą stroną do Słońca.
\begin{enumerate}
\item Znajdź wyrażenie na temperaturę sondy w funkcji jej odległości od Słońca i porównaj z temperaturą sondy
pozbawionej osłony
\item Dobierz współczynnik odbicia $r$ tak aby w pobliżu Merkurego sonda miała temperaturę $T_{\rm sondy}= 300$\,K. 
Jaka jest wtedy temperatura osłony?
\end{enumerate}
Temperatura Słońca wynosi $T_\odot= 5800$\,K, promień Słońca $R_\odot = 7 \cdot 10^8$\,m, 
odległość Merkurego od Słońca $d = 5.8 \cdot 10^{10}$\,m. 
Zakładamy, że cała powierzchnia sondy ma tę samą temperaturę.

\vskip 10pt
\textbf{Odpowiedź:}
\begin{enumerate}
\item Bez osłony: $T_{sondy}^4 = \frac{T_\odot^4}{4} \frac{R_\odot^2}{d^2}$
Z osłoną: $T_{sondy}^4 = \frac{1-r}{3(2-r)} \frac{T_\odot^4}{4} \frac{R_\odot^2}{d^2}$
\item $r \approx 0.755$
\end{enumerate}

\end{document}

\zadanie
Opór właściwy półprzewodnika zależy od temperatury jak:
$ \rho(T) = A \exp{(\alpha / T)}, $ 
gdzie $\alpha = 0.01$eV/k$_B$.
Jakie zmiany temperatury  w okolicy 290\,K można mierzyć takim termometrem
zakładając, że potrafimy mierzyć opór z dokładnością do 0.01\%.
Przyjmij k$_B = 8.62 \cdot 10^{-5}$ eV/K.