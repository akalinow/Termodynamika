\documentclass[11pt,a4paper]{article}

\usepackage[polish]{babel}
\usepackage[utf8]{inputenc}
\usepackage{polski}
\usepackage[T1]{fontenc}
\usepackage{indentfirst}
\usepackage{wrapfig}    % for wrapping figures, tables

\frenchspacing

\usepackage{amsmath}
%\usepackage{physics}
%\usepackage{bm}
\usepackage{gensymb}
%\usepackage{hepnames}
\usepackage{epsfig}
\usepackage{graphics}
\usepackage[shortlabels]{enumitem}
%\usepackage{xspace}
%\xspaceaddexceptions{[]\{\}}

%
%
%fixpagesize
\pagestyle{empty}
\addtolength{\textwidth}{6cm}
\addtolength{\textheight}{4cm}
\addtolength{\evensidemargin}{-3cm}
\addtolength{\oddsidemargin}{-3cm}
\addtolength{\topmargin}{-2cm}
\parindent=0cm


%
%
%small distance in list/item/enum for enumitem package
\setlist[itemize,enumerate]{topsep=0em}
\setlist{noitemsep}

%print zadanie #
\newcounter{zadanie}\newcommand{\zadanie}[1][]{\addtocounter{zadanie}{1} ~\\  {\bf \emph{Zadanie \arabic{zadanie} #1 }} \\}
\newcounter{zaddom}\newcommand{\zaddom}[1][]{\addtocounter{zaddom}{1} ~\\  {\bf \emph{Zadanie domowe \arabic{zaddom} #1 }} \\}
%\renewcommand{\zadanie}[1][]{\pagebreak  ~\\  {\bf \emph{Zadanie }} \\} \addtolength{\topmargin}{-2cm}

\newcommand{\dbar}{{\mkern3mu\mathchar'26\mkern-12mu d}}
\newcommand{\Partial}[3]{\left( \frac{\partial #1}{\partial #2} \right)_{#3}}


%%%%%%%%%%%%%%%%%%%%%%%%%%%%%%%%%%%%%%%%%%%%%%%%%%%%%%
\begin{document}           % End of preamble and beginning of text.

\begin{centering}
\bf{\Large{Termodynamika z elementami fizyki statystycznej}}\\
Tydzień 4  (23 marca 2023)\\[3mm]
Równanie stanu, rozwinięcia wirialne, parametry krytyczne \\ 
\end{centering} 
\vspace{5mm}

\zadanie
Pokaż, że: \[ 
\left(\frac{\partial p}{\partial T}\right)_V \cdot
\left(\frac{\partial T}{\partial V}\right)_p \cdot
\left(\frac{\partial V}{\partial p}\right)_T = -1 
\]
\\
\\
\textbf{Rozwiązanie:}
\\
Zakładamy, że słuszne jest ogólne równanie stanu postaci:
\begin{equation*}
    f(p,V,T)=0.
\end{equation*}
Wynika stąd, że dowolny parametr stanu np. ciśnienie możemy wyrazić jako funkcję dwóch pozostałych parametrów, tu: $T,V$
\begin{equation*}
    p=p(T,V).
\end{equation*}
W ogólności
\begin{equation*}
    dp = \Partial{p}{T}{V}dT+\Partial{p}{V}{T}dV.
\end{equation*}
Rozważmy infinitezymalny proces pod stałym ciśnieniem $p=const, \, dp=0$. Mamy
\begin{equation}
\label{eq0}
    0=\Partial{p}{T}{V}dT+\Partial{p}{V}{T}dV.
\end{equation}
Podczas takiego procesu
\begin{equation*}
    dV=\Partial{V}{T}{p}dT.
\end{equation*}
wstawiając takie wyrażenie na $dV$ do \eqref{eq0} otrzymujemy relację
\begin{equation*}
    0=\Partial{p}{T}{V}+\Partial{p}{V}{T}\Partial{V}{T}{p}
\end{equation*}
Zatem
\begin{equation*}
    \Partial{p}{T}{V}=-\Partial{p}{V}{T}\Partial{V}{T}{p}
\end{equation*}
Korzystając z wzoru na pochodną funkcji odwrotnej dostajemy ostatecznie
\[ 
\left(\frac{\partial p}{\partial T}\right)_V \cdot
\left(\frac{\partial T}{\partial V}\right)_p \cdot
\left(\frac{\partial V}{\partial p}\right)_T = -1.
\]

\newpage


\zadanie
Pokaż, że zachodzi związek $\gamma=p\beta\kappa$, gdzie $p$ - ciśnienie, 
$\gamma$ - współczynnik rozszerzalności temperaturowej,
$\beta$ - współczynnik temperaturowej zmiany ciśnienia, 
$\kappa$ - współczynnik ściśliwości izotermicznej. 
Wynik sprawdź dla gazu doskonałego (korzystając z wyników z poprzednich ćwiczeń).\\

{\it Wskazówka:} Skorzystaj z wyniku poprzedniego zadania.\\
\\
\textbf{Rozwiązanie:}
\\
Definicje współczynników $\gamma, \beta, \kappa$ są następujące
\begin{equation*}
    \gamma=\frac{1}{V} \Partial{V}{T}{p} \qquad \beta= \frac{1}{p} \Partial{p}{T}{V} \qquad \kappa =-\frac{1}{V} \Partial{V}{p}{T}.
\end{equation*}
Z tożsamości wyprowadzonej w poprzednim zadaniu i z wzoru na pochodną funkcji odwrotnej znajdujemy
\begin{equation*}
    \Partial{V}{T}{p}=-\Partial{V}{p}{T}\Partial{p}{T}{V}.
\end{equation*}
Wobec czego 
\begin{equation*}
    \gamma=\frac{1}{V} \Partial{V}{T}{p}=-\frac{1}{V}\Partial{V}{p}{T}\Partial{p}{T}{V}=\kappa p \beta.
\end{equation*}
Sprawdźmy teraz dla gazu doskonałego. Na poprzednich ćwiczeniach znaleźliśmy 
\begin{equation*}
    \gamma = \frac{1}{T} \qquad \kappa=\frac{1}{p} \qquad \beta =\frac{1}{T}.
\end{equation*}
Widzmy, że dla takich współczynników $\gamma =p \beta \kappa$ jest spełnione w oczywisty sposób.
\newpage

\zadanie
Rozpatrzmy rtęć wypełniającą zbiornik w warunkach normalnych. 
Jak zmieni się ciśnienie wywierane na ścianki przy ogrzaniu o 100\degree C?
Znajdź wynik, wiedząc że 
$\gamma=181\cdot 10^{-6}$\,K$^{-1}$, $\kappa= 3.82\cdot 10^{-11}$\,Pa$^{-1}$ słabo zależą od
temperatury i można je przyjąć jako stałe.\\

\textbf{Rozwiązanie:}
\\
Rtęć nie zmienia swojej objętości podczas podgrzewania, piszemy więc
\begin{equation}
\label{eq1}
    p(T+\Delta T)=p(T)+ \Partial{p}{T}{V} \Delta T +O(\Delta T^2).
\end{equation}
Z definicji współczynnika $\beta$ mamy
\begin{equation*}
    \beta p =\Partial{p}{T}{V}
\end{equation*}
jednocześnie, korzystając z tożsamości z poprzedniego zadania znajdujemy
\begin{equation*}
    \Partial{p}{T}{V} = \beta p =\frac{\gamma}{\kappa}
\end{equation*}
zarówno $\gamma$ jak i $\kappa$ przyjmujemy jako niezależne od temperatury, więc i $\Partial{p}{T}{V}$ nie zależy od $T$. Wobec czego, wyrazy z wyższymi potęgami w $\Delta T$ w równaniu \eqref{eq1} możemy pominąć.
Znajdujemy
\begin{equation*}
    \Delta p = p(T+\Delta T)-p(T)=\frac{\gamma}{\kappa} \Delta T \approx 4738,22 \text{Ba}.
\end{equation*}
\newpage

\zadanie
Znaleźć równanie stanu gazu, jeżeli znane są następujące fakty:
\begin{enumerate}
\item w temperaturze $T$ i ciśnieniu $p=1$ spełniony jest związek
$\gamma V = R,$
\item w temperaturze $T$ i dowolnym ciśnieniu $p$ słuszny jest wzór $\kappa V = \frac{RT}{p^2}-\frac{2a}{p^3},$
gdzie $a$ jest pewną znaną stałą,
\item $\lim_{p\rightarrow\infty}V=V_0,$
\end{enumerate}
przy czym $\gamma=\frac{1}{V}\left(\frac{\partial V}{\partial T}\right)_p$ to współczynnik rozszerzalności temperaturowej, a
\mbox{$\kappa=-\frac{1}{V}\left(\frac{\partial V}{\partial p}\right)_T$} jest współczynnikiem ściśliwości izotermicznej.\\
\\
\textbf{Rozwiązanie:}
\\
Z definicji współczynnika ściśliwości iotermicznej mamy:
\begin{equation}
    \kappa V = -\Partial{V}{p}{T}
\end{equation}
korzystając z faktu 2. piszemy
\begin{equation*}
    -\Partial{V}{p}{T}=\frac{RT}{p^2}-\frac{2a}{p^3}
\end{equation*}
wiemy, że $V=V(p,T)$. Całkując powyższe równanie po ciśnieniu otrzymujemy
\begin{equation}
\label{eq2}
    V = \frac{RT}{p}-\frac{a}{p^2}+f(T)
\end{equation}
gdzie $f(T)$ jest pewną funkcją temperatury, którą teraz wyznaczymy. Z definicji współczynnika rozszerzalności temperaturowej
\begin{equation*}
    \gamma V = \Partial{V}{T}{p}=\frac{R}{p}+\frac{ df}{dT}
\end{equation*}
Z faktu 1. mamy
\begin{equation*}
    \left(\frac{R}{p}+\frac{ df}{dT}\right )_{p=1}=R+\frac{ df}{dT}=R
\end{equation*}
stąd wniosek
\begin{equation*}
    f(T)=const. \equiv f
\end{equation*}
z faktu 3. i równania \eqref{eq2} znajdujemy, że $f=V_0$. Ostatecznie
\begin{equation*}
    V(p,T)=\frac{RT}{p}-\frac{a}{p^2}+V_0.
\end{equation*}
\newpage
\zadanie
Znajdź ogólną postać równania stanu substancji, której współczynniki rozszerzalności temperaturowej  $\gamma$ i izotermicznej ściśliwości $\kappa$ spełniają równania:
\[ \gamma=\frac{a T^2}{p}, ~~~\kappa=\frac{b T^3}{p^2}. \]
Wyznacz stosunek $a/b$.\\
\\
\textbf{Rozwiązanie:}
Z definicji współczynników $\gamma$ i $\beta$ dostajemy
\begin{equation}
\label{eq3}
    \gamma=\frac{1}{V}\Partial{V}{T}{p}=\frac{a T^2}{p}
\end{equation}
oraz
\begin{equation}
\label{eq4}
    \beta=-\frac{1}{V}\Partial{V}{p}{T}=\frac{bT^3}{p^2}.
\end{equation}
Wiemy, że $V=V(p,T)$. Rozważając \eqref{eq3} dla stałego ciśnienia możemy napisać równanie
\begin{equation*}
    \frac{dV}{V}=\frac{aT^2}{p}dT.
\end{equation*}
Analogicznie, rozważając \eqref{eq4} dla stałej temperatury znajdujemy
\begin{equation*}
    \frac{dV}{V}=-\frac{bT^3}{p^2}dp.
\end{equation*}
Odcałkowanie tych równań daje odpowiednio
\begin{equation*}
    \ln V(p,T)=\frac{aT^3}{3p}+f(p)
\end{equation*}
i
\begin{equation*}
    \ln V(p,T)=\frac{bT^3}{p}+\tilde{f}(T)
\end{equation*}
gdzie $f(p)$ zależy tylko od ciśnienia, a $\tilde{f}(T)$ zależy tylko od temperatury. Aby uzyskać zgodność obu powyższych zależności musimy mieć
\begin{equation*}
    f(p)=\tilde{f}(T)=const,
\end{equation*}
ponadto
\begin{equation*}
    \frac{a}{3}=b.
\end{equation*}
Ostatecznie, ogólna postać równania stanu to
\begin{equation*}
    V(p,T)=C\exp \left( \frac{a T^3}{3p} \right)
\end{equation*}
i
\begin{equation*}
    a/b=3.
\end{equation*}

\newpage
\zadanie
Znaleźć dwa pierwsze wyrazy rozwinięcia wirialnego 
$\displaystyle \frac{p v}{R T} = 1 + \frac{B(T)}{v} + \frac{C(T)}{v^2} + \ldots$ \\
dla równania stanu van der Waalsa
      $\displaystyle p(T,v) = \frac{R T}{v-b} - \frac{a}{v^2}$, \\
{\em Uwaga:} $v$ -- objętość molowa. 
Aby zapisać równania dla $n$ moli, trzeba podstawić $v \mapsto V/n$.\\
\\
\textbf{Rozwiązanie:}
\\
Podaną postać równania stanu mnożymy przez $\frac{v}{RT}$ otrzymując
\begin{equation*}
    \frac{pv}{RT}=\frac{v}{v-b}-\frac{a}{RTv}.
\end{equation*}
Chcemy zapisać prawą stronę jako szereg w $1/v$, mamy
\begin{equation*}
    \frac{v}{v-b}=\frac{1}{1-b/v}=1+\frac{b}{v}+\frac{b^2}{v^2}+\ldots
\end{equation*}
wobec czego
\begin{equation*}
    \frac{pv}{RT}=1+\left(b-\frac{a}{RT} \right) \frac{1}{v}+\frac{b^2}{v^2}+\ldots.
\end{equation*}
Stąd
\begin{equation*}
    B(T)=b-\frac{a}{RT} \qquad C(T)=b^2.
\end{equation*}


\newpage
\zadanie
Znaleźć współrzędne $p_k, v_k$ i $T_k$ punktu krytycznego
gazu van der Waalsa oraz tzw. krytyczny
współczynnik kompresji $\displaystyle z_k =\frac{p_k v_k}{R T_k}$,
a następnie przedstawić równanie van der Waalsa w
zmiennych zredukowanych: $\pi=p/p_k$, $\omega=v/v_k$ i $\tau=T/T_k$.\\

{\it Uwaga:} W punkcie krytycznym rówanie stanu $f(p,v,T)=0$ ma pierwiatek potrójny oraz występuje punkt przegięcia izotermy.\\

{\it Wskazówka 1:} Przekształć równanie stanu gazu Van der Waalsa do postaci wielomianu w $v$.\\

{\it Wskazówka 2:} Istnieje więcej niż jeden sposób rozwiązania tego zadania.\\
\\
\textbf{Rozwiązanie:}
\\
\textit{Sposób 1.}
\\
Rozważmy izotermy Van der Waalsa. Znajdźmy temperaturę odpowiadającą izotermie, która wykazuje punkt przegięcia. W tym punkcie takim spełnione są następujące warunki:
\begin{itemize}
    \item znikanie pierwszej pochodnej
    \begin{equation*}
        \Partial{p}{v}{T}=0, \text{w punkcie } (p_k,v_k,T_k)
    \end{equation*}
    \item znikanie drugiej pochodnej
    \begin{equation*}
    \left( \frac{\partial^2 p}{\partial v^2} \right)_T=0, \text{w punkcie } (p_k,v_k,T_k)
    \end{equation*}
    \item spełnione jest równanie stanu
    \begin{equation*}
        f(p_k,v_k,T_k)=0.
    \end{equation*}
\end{itemize}
Problem sprowadza się do rozwiązania układu równań
\begin{equation*}
    RT_kv_k^3=2a(v_k-b)^2
\end{equation*}
\begin{equation*}
    RT_kv_k^4=3a(v_k-b)^3
\end{equation*}
\begin{equation*}
    p_k=\frac{RT_k}{v_k-b}-\frac{a}{v_k^2}.
\end{equation*}
Z pierwszych dwóch równań znajdujemy
\begin{equation*}
    v_k=3b \qquad T_k=\frac{8}{27}\frac{a}{bR}
\end{equation*}
podstawiając te wartości do trzeciego znajdujemy
\begin{equation*}
    p_k=\frac{1}{27} \frac{a}{b^2}.
\end{equation*}
Ostatecznie 
\begin{equation*}
    z_k=\frac{p_k v_k}{R T_k}=\frac{3}{8}
\end{equation*}
\textit{Sposób 2.}
\\
Na początku sprowadźmy równanie van der Waalsa do postaci wielomianu zmiennej $v$. Wiemy, że
\begin{equation*}
    p=\frac{RT}{v-b}-\frac{a}{v^2}
\end{equation*}
\begin{equation*}
    (v-b)v^2p=RTv^2-a(v-b)
\end{equation*}
\begin{equation*}
    v^3-\left( b+\frac{RT}{p} \right)v^2+\frac{a}{p}v-\frac{ab}{p}=0
\end{equation*}
Wiemy, że w punkcie krytycznym istnieje jeden pierwiastek potrójny. Powyższe równanie dla $p_k, T_k$ powinno dać więc się przedstawić w postaci $(v-v_k)^3=0$. Piszemy więc
\begin{equation*}
    (v-v_k)^3=v^3-\left( b+\frac{RT_k}{p_k} \right)v_k^2+\frac{a}{p_k}v-\frac{ab}{p_k}
\end{equation*}
Przyrównując współczynniki przy odpowiednich potęgach $v$ znajdujemy
\begin{equation*}
    3v_k=b+\frac{RT_k}{p_k}
\end{equation*}
\begin{equation*}
    3v_k^2=\frac{a}{p_k}
\end{equation*}
\begin{equation*}
    v_k^3=\frac{ab}{p_k}
\end{equation*}
Dzieląc trzecie równanie przez drugie znajdujemy 
\begin{equation*}
    v_k=3b
\end{equation*}
a stąd
\begin{equation*}
    p_k=\frac{1}{27}\frac{a}{b^2} \qquad T_k=\frac{8}{27}\frac{a}{bR}
\end{equation*}
Na koniec, wyraźmy równanie stanu w zmiennych zredukowanych $\pi =p/p_k$, $\omega=v/v_k$ i $\tau=T/T_k$. Zastępując zmienne $p,V,T$ zmiennymi $\pi p_k, \omega v_k, \tau T_k$ znajdujemy
\begin{equation*}
    \pi p_k=\frac{R \tau T_k}{\omega v_k-b}-\frac{a}{\omega^2 v_k^2},
\end{equation*}
\begin{equation*}
    \frac{\pi}{27}\frac{a}{b^2}=\frac{\tau \frac{8}{27}\frac{a}{b}}{b(3\omega-1)}-\frac{a}{9\omega^2 b^2}.
\end{equation*}
I ostatecznie
\begin{equation*}
    (3 \omega-1) \left( \pi +\frac{3}{\omega^2} \right)=8 \tau.
\end{equation*}


% An example of figure placement:
%\begin{wrapfigure}[13]{r}{0.4\linewidth}\vspace{3mm}
%\resizebox{\linewidth}{!}{\includegraphics{NAZWA.png}}
%\end{wrapfigure}
%\zadanie

\end{document}
