\documentclass[11pt,a4paper]{article}

\usepackage[polish]{babel}
\usepackage[utf8]{inputenc}
\usepackage{polski}
\usepackage[T1]{fontenc}
\usepackage{indentfirst}
\usepackage{wrapfig}    % for wrapping figures, tables

\frenchspacing

\usepackage{amsmath}
%\usepackage{physics}
%\usepackage{bm}
\usepackage{gensymb}
%\usepackage{hepnames}
\usepackage{epsfig}
\usepackage{graphics}
\usepackage[shortlabels]{enumitem}
%\usepackage{xspace}
%\xspaceaddexceptions{[]\{\}}

%
%
%fixpagesize
\pagestyle{empty}
\addtolength{\textwidth}{6cm}
\addtolength{\textheight}{4cm}
\addtolength{\evensidemargin}{-3cm}
\addtolength{\oddsidemargin}{-3cm}
\addtolength{\topmargin}{-2cm}
\parindent=0cm


%
%
%small distance in list/item/enum for enumitem package
\setlist[itemize,enumerate]{topsep=0em}
\setlist{noitemsep}

%print zadanie #
\newcounter{zadanie}\newcommand{\zadanie}[1][]{\addtocounter{zadanie}{1} ~\\  {\bf \emph{Zadanie \arabic{zadanie} #1 }} \\}
\newcounter{zaddom}\newcommand{\zaddom}[1][]{\addtocounter{zaddom}{1} ~\\  {\bf \emph{Zadanie domowe \arabic{zaddom} #1 }} \\}
%\renewcommand{\zadanie}[1][]{\pagebreak  ~\\  {\bf \emph{Zadanie }} \\} \addtolength{\topmargin}{-2cm}

\newcommand{\dbar}{{\mkern3mu\mathchar'26\mkern-12mu d}}


%%%%%%%%%%%%%%%%%%%%%%%%%%%%%%%%%%%%%%%%%%%%%%%%%%%%%%
\begin{document}           % End of preamble and beginning of text.

\begin{centering}
\bf{\Large{Termodynamika z elementami fizyki statystycznej}}\\
Tydzień 4  (23 marca 2023)\\[3mm]
Równanie stanu, rozwinięcia wirialne, parametry krytyczne \\ 
\end{centering} 
\vspace{5mm}

\zadanie
Pokaż, że: \[ 
\left(\frac{\partial p}{\partial T}\right)_V \cdot
\left(\frac{\partial T}{\partial V}\right)_p \cdot
\left(\frac{\partial V}{\partial p}\right)_T = -1 
\]

\zadanie
Pokaż, że $\gamma=p\beta\kappa$, gdzie $p$ - ciśnienie, 
$\gamma$ - współczynnik rozszerzalności temperaturowej,
$\beta$ - współczynnik temperaturowej zmiany ciśnienia, 
$\kappa$ - współczynnik ściśliwości izotermicznej. 
Wynik sprawdź dla gazu doskonałego (korzystając z wyników z poprzednich ćwiczeń).\\

{\it Wskazówka:} Skorzystaj z wyniku poprzedniego zadania.\\


\zadanie
Rozpatrzmy rtęć wypełniającą zbiornik w warunkach normalnych. 
Jak zmieni się ciśnienie wywierane na ścianki przy ogrzaniu o 100\degree C?
Znajdź wynik, wiedząc że 
$\gamma=181\cdot 10^{-6}$\,K$^{-1}$, $\kappa= 3.82\cdot 10^{-11}$\,Pa$^{-1}$ słabo zależą od
temperatury i można je przyjąć jako stałe.\\


\zadanie
Znaleźć równanie stanu gazu, jeżeli znane są następujące fakty:
\begin{enumerate}
\item w temperaturze $T$ i ciśnieniu $p=1$ spełniony jest związek
$\gamma V = R,$
\item w temperaturze $T$ i dowolnym ciśnieniu $p$ słuszny jest wzór $\kappa V = \frac{RT}{p^2}-\frac{2a}{p^3},$
gdzie $a$ jest pewną znaną stałą,
\item $\lim_{p\rightarrow\infty}V=V_0,$
\end{enumerate}
przy czym $\gamma=\frac{1}{V}\left(\frac{\partial V}{\partial T}\right)_p$ to współczynnik rozszerzalności temperaturowej, a
\mbox{$\kappa=-\frac{1}{V}\left(\frac{\partial V}{\partial p}\right)_T$} jest współczynnikiem ściśliwości izotermicznej.\\


\zadanie
Znajdź ogólną postać równania stanu substancji, której współczynniki rozszerzalności temperaturowej  $\gamma$ i izotermicznej ściśliwości $\kappa$ spełniają równania:
\[ \gamma=\frac{a T^2}{p}, ~~~\kappa=\frac{b T^3}{p^2}. \]
Wyznacz stosunek $a/b$.\\


\zadanie
Znaleźć dwa pierwsze wyrazy rozwinięcia wirialnego 
$\displaystyle \frac{p v}{R T} = 1 + \frac{B(T)}{v} + \frac{C(T)}{v^2} + \ldots$ \\
dla równania stanu van der Waalsa
      $\displaystyle p(T,v) = \frac{R T}{v-b} - \frac{a}{v^2}$, \\
{\em Uwaga:} $v$ -- objętość molowa. 
Aby zapisać równania dla $n$ moli, trzeba podstawić $v \mapsto V/n$.\\


\zadanie
Znaleźć współrzędne $p_k, v_k$ i $T_k$ punktu krytycznego
gazu van der Waalsa oraz tzw. krytyczny
współczynnik kompresji $\displaystyle z_k =\frac{p_k v_k}{R T_k}$,
a następnie przedstawić równanie van der Waalsa w
zmiennych zredukowanych: $\pi=p/p_k$, $\omega=v/v_k$ i $\tau=T/T_k$.\\

{\it Uwaga:} W punkcie krytycznym rówanie stanu $f(p,v,T)=0$ ma pierwiatek potrójny oraz występuje punkt przegięcia izotermy.\\

\pagebreak
\zaddom
Znając dla gazu doskonałego $\gamma$, $\beta$, $\kappa$, odtwórz r. stanu.

\zaddom
Dany jest gaz, którego zachowanie w pewnym zakresie parametrów opisuje równanie stanu:
\[ p V = R T \cdot\exp\left[ - \frac{a}{T V} \right], {\rm ~~~gdzie~ }R,a = const. \]
Znajdź współczynniki: rozszerzalności temperaturowej
$\displaystyle \gamma \equiv \frac{1}{V}\left(\frac{\partial V}{\partial T}\right)_p$
oraz izotermicznej ściśliwości
\mbox{$\displaystyle \kappa \equiv -\frac{1}{V}\left(\frac{\partial V}{\partial p}\right)_T$}
dla tego gazu.\\

{\it Odpowiedź:}
\[
  \begin{array}{c}
    \gamma = \frac{V+a/T}{VT-a}\\
    \\
    \kappa = \frac{1}{p}\frac{VT}{VT-a}
  \end{array}
\]

\zaddom
Dany jest gaz, którego zachowanie w pewnym zakresie parametrów opisuje równanie stanu:
\[  p^3 \cdot T^{-1/2} \cdot \exp( \alpha V ) = c = const.\]
Znajdź współczynnki izobarycznej rozszerzalności temperaturowej oraz izotermicznej ściśliwości poprzez:
\begin{enumerate}[a)]
\item znalezienie odpowiednich pochodnych cząstkowych po wyznaczeniu $V$ z równiania stanu
\item licząc różniczkę równania stanu, a następnie wyrażając $dV$ przez $dT$ i $dp$.
\end{enumerate}

{\it Odpowiedź:}
\[
  \begin{array}{c}
    \gamma = \left(2T \ln\left( \frac{c \sqrt{T}}{p^3} \right) \right)^{-1}\\
    \\
    \kappa = \frac{3}{p} \left( \ln\left( \frac{c \sqrt{T}}{p^3} \right) \right)^{-1}
  \end{array}
\]    

\zaddom
Znaleźć dwa pierwsze wyrazy rozwinięcia wirialnego w $1/v$ dla gazu Dietericiego: 
$\displaystyle p(T,v) = \frac{R T}{v-b}\cdot\exp\left[-\frac{a}{R T v}\right]$.\\

{\em Uwaga: Aby zapisać te równania dla $n$ moli, trzeba podstawić $v\mapsto V/n$}.\\

{\it Odpowiedź:}
\[
  \frac{pv}{RT} = 1 + \left(b - \frac{a}{RT} \right) \frac{1}{v} + \left( b^2 - \frac{b a}{R T} + \frac{a^2}{2 R^2 T^2} \right) \left( \frac{1}{v} \right)^2 + ...
\]

\zaddom
Wiadomo, że równanie van der Waalsa daje współczynnik krytyczny $K$
(gdzie $K\equiv 1/z_k$), mniejszy od wartości doświadczalnej.
Jaki współczynnik krytyczny dawałoby zmodyfikowane równanie
postaci
\linebreak
\mbox{$\displaystyle p(T,V) = \frac{R T}{V-b} - \frac{a}{T V^n}$}? \\
Ile powinien wynosić parametr $n$, aby otrzymać  wartość
$K = 3.8$ ?
Ile razy objętość krytyczna $V_k$
byłaby większa od parametru $b$? \\

{\em Uwaga: Dla $n=2$ powyższe równanie nosi nazwę równania Berthelota.}\\

{\it Odpowiedź:} $n=1.656$; $V_k \approx 4b$


% An example of figure placement:
%\begin{wrapfigure}[13]{r}{0.4\linewidth}\vspace{3mm}
%\resizebox{\linewidth}{!}{\includegraphics{NAZWA.png}}
%\end{wrapfigure}
%\zadanie

\end{document}
