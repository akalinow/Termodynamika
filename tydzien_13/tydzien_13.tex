\documentclass[11pt,a4paper]{article}

\usepackage[polish]{babel}
\usepackage[utf8]{inputenc}
\usepackage{polski}
\usepackage[T1]{fontenc}
\usepackage{indentfirst}
\usepackage{wrapfig}    % for wrapping figures, tables

\frenchspacing

%\usepackage{amsmath}
\usepackage{physics}
%\usepackage{bm}
\usepackage{gensymb}
%\usepackage{hepnames}
\usepackage{epsfig}
\usepackage{graphics}
\usepackage[shortlabels]{enumitem}
%\usepackage{xspace}
%\xspaceaddexceptions{[]\{\}}

%
%
%fixpagesize
\pagestyle{empty}
\addtolength{\textwidth}{6cm}
\addtolength{\textheight}{4cm}
\addtolength{\evensidemargin}{-3cm}
\addtolength{\oddsidemargin}{-3cm}
\addtolength{\topmargin}{-2cm}
\parindent=0cm

%
%
%small distance in list/item/enum for enumitem package
\setlist[itemize,enumerate]{topsep=0em}
\setlist{noitemsep}


% definition of inexact differential symbol:
\newcommand{\dbar} {\ensuremath{\,\mathchar'26\mkern-12mu d}}

%print zadanie #
\newcounter{zadanie}\newcommand{\zadanie}[1][]{\addtocounter{zadanie}{1} ~\\  {\bf \emph{Zadanie \arabic{zadanie} #1 }} \\}
\newcounter{zaddom}\newcommand{\zaddom}[1][]{\addtocounter{zaddom}{1} ~\\  {\bf \emph{Zadanie domowe \arabic{zaddom} #1 }} \\}
%\renewcommand{\zadanie}[1][]{\pagebreak  ~\\  {\bf \emph{Zadanie }} \\} \addtolength{\topmargin}{-2cm}


%
%%%%%%%%%%%%%%%%%%%%%%%%%%%%%%%%%%%%%%%%%%%%%%%%%%%%%%
% Changes figure placing algorithm
\renewcommand{\topfraction}{1}       % maximal fraction of a page allowed for figures
\renewcommand{\textfraction}{0.15}   % minimal number of text for figure-text shared pages
\renewcommand{\floatpagefraction}{0.95} % if two above does not help, this could do the job
                                        % must be: floatpagefraction < topfraction !!!!
%
\renewcommand{\textfraction}{0} % minimum fraction of page, which must be
                                % devoted to text
\renewcommand{\topfraction}{1}  % maximum fraction at top, which can be
                                % occupied whit floats
\setcounter{totalnumber}{400}   % increase the number of floats for one page
\setcounter{topnumber}{200}     % at all/top/bottom.
\setcounter{bottomnumber}{200}  %


\begin{document}           % End of preamble and beginning of text.
\begin{centering}
\bf{\Large{Termodynamika z elementami fizyki statystycznej}}\\
Tydzień 13 (24 maja 2023)\\[5mm]
Ekwipartycja energii, mikroskopowy obraz ciśnienia, równanie równowagi hydrostatycznej\\
\end{centering}
\vspace{10mm}


\zadanie
Rozważmy jednoatomowy gaz doskonały w temperaturze $T$ i w stanie równowagi termodynamicznej.
Obliczyć średnią energię kinetyczną cząsteczki (atomu) tego gazu.
Zastanowić się nad powodem spełnienia zasady ekwipartycji.
\newline

%\emph{Wskazówka:} Rozważ rozkład Boltzmana parametryzowany prędkością cząstki, gdzie $E_{\vec{v}} = \frac{m}{2}(v_x^2+v_y^2+v_z^2)$.
%\newline

%\emph{Odpowiedź:} $\langle E \rangle = \frac{3}{2} k T$.


\zadanie
{\em Przykład, w którym zasada ekwipartycji nie działa}\\
Cząstka o masie $m$ porusza się ruchem jednowymiarowym w polu o energii potencjalnej
$E_p(x) = A \cdot |x|^n$, gdzie $A>0$, $n\ge 1$.
Układ jest w kontakcie z termostatem o temperaturze $T$.
Obliczyć średnie energie cząstki: potencjalną, kinetyczną oraz całkowitą.
\newline

%\emph{Wskazówka:} Udowodnij a następnie skorzystaj z następującej tożsamości:
%$\int_{-\infty}^\infty  e^{- \alpha |x|^n}\textrm{d}x = \alpha n \int_{-\infty}^\infty |x|^n e^{-\alpha |x|^n} \textrm{d}x$.
%\newline

%\emph{Odpowiedź:} $\langle E \rangle = (\frac{1}{2} + \frac{1}{n}) k T$. Czyli zasada ekwipartycji spełniona jedynie dla $n=2$.


\zadanie
Jako przykład rozkładu ciągłego zmiennej losowej rozważmy
rozkład prędkości cząsteczek gazu\linebreak doskonałego w temperaturze $T$
(tzw. rozkład Maxwella):
\[ D(v) = \left(\frac{m}{2\pi k T}\right)^{\frac{3}{2}}
          \cdot 4 \pi v^2 \cdot\exp\left[ -\frac{m v^2}{2 k T} \right],\]
gdzie: $m$ jest masą cząsteczki, $k$ – stałą Boltzmanna, $T$ – temperaturą bezwzględną, zaś $v$ – wartością prędkości cząsteczki.
$D(v)$ jest gęstością prawdopodobieństwa, tzn. $D(v)\cdot dv = P(v, v+dv)$ jest prawdopodobieństwem tego, że prędkość cząsteczki ma wartość zawartą między
$v$ a $v+dv$. Policzyć najbardziej prawdopodobną oraz średnią wartość prędkości. Wykonać obliczenia dla cząsteczki azotu o masie $m = 28\,$u ($1\,{\rm u} = 1.66\cdot 10^{-27}\,$kg) w temperaturze $T = 300\,$K.
Stała Boltzmanna wynosi $k = 1.38\cdot 10^{-23}\,$J/K.
\newline

%\emph{Odpowiedź:} Wartość najbardziej prawdopodobna $v_{\textrm{max}} = \sqrt{\frac{2 k T}{m}} = 429\, \textrm{m}/\textrm{s}$.
%Wartość średnia $\langle v \rangle = \frac{2}{\sqrt{\pi}} v_{\textrm{max}} = 485\, \textrm{m}/\textrm{s}$.

\zadanie
Rozważyć pionową nieskończoną kolumnę gazu złożoną z cząsteczek o masie $m$,
utrzymywaną w stałej temperaturze $T$, znajdującą się w jednorodnym polu grawitacyjnym
o natężeniu $g$. Wyznaczyć wysokość środka masy gazu posługując się rozkładem kanonicznym.
\newline

%\emph{Odpowiedź:} $\langle x \rangle = \frac{k T }{m g}$

\zadanie
Rozwiąż zadanie poprzednie korzystając z równania równowagi hydrosatycznej
\newline

%\emph{Wskazówka:} Rozważamy cienkie warstwy powietrza i obliczamy jak musi sie zmieniać ciśnienie a w konsekwencji gęstość gazu (zgodnie z równaniem gazy doskonałego) aby spełniony był warunek równowagi hydrostatycznej. Położenie środka masy wyjdzie takie samo $\langle x \rangle = \frac{k T }{m g}$.

\zadanie
{\em Przypadek stałej gęstości, ale niejednorodnego pola grawitacyjnego.} \\
Oszacować ciśnienie panujące w centrum Ziemi.
Dane są: promień Ziemi $R_Z = 6370$\,km i masa Ziemi \linebreak \mbox{$M_Z = 5.98\cdot10^{24}$\,kg} oraz wartość
przyspieszenia ziemskiego na powierzchni Ziemi $g\approx 10$\,m/s$^2$.
Zakładamy, że Ziemia jest jednorodną kulą.

%\emph{Odpowiedź:} $p =  176\,\textrm{GPa}$ (faktyczne ciśnienie wynosi ok. $300 \textrm{GPa}$).
\newpage

\zadanie
{\em Przypadek zmiennej gęstości i jednorodnego pola grawitacyjnego.} \\
Jaki jest wpływ zmiany temperatury na ciśnienie atmosferyczne?
Rozwiązać równanie barometryczne zakładając, że dla wysokości w zakresie $0-10$\,km:
$g=const$, powietrze spełnia równanie gazu doskonałego,
temperatura maleje liniowo z wysokością, o 5\,K na 1\,km.
Wynik porównać z~przypadkiem stałej temperatury.
Przeprowadzić przejście graniczne do stałej temperatury.
\newline

%\emph{Odpowiedź:} Zakładamy, że $T(x) = T_0 - \alpha x$. Wtedy $p(x) = p_0\left(1- \frac{\alpha x}{T_0}\right)^{\frac{m g}{\alpha k}}$.
%Aby pokazać, przejście graniczne do wzoru barometrycznego dla stałej temperatury ($\alpha \rightarrow 0$) należy skorzystać z $\lim_{n\rightarrow \infty} (1+x/n)^n =e^x$.

\vspace{2cm}

%\pagebreak
%
% domowe
%
\zaddom
Energia kwantowego oscylatora harmonicznego wynosi $E(n)=(n+\frac{1}{2})\hbar\omega$,
gdzie $n = 0,1\ldots\infty$.
Znajdź średnią energię oscylatora w otoczeniu o temperaturze $T$.
Jaki jest wkład oscylacji do ciepła właściwego?
Przedyskutuj postać rozwiązania w zakresie niskich i wysokich
temperatur (odpowiednio $k T \ll \hbar\omega$ i $k T \gg \hbar\omega$).
Porównaj wynik z oscylatorem klasycznym.

\emph{Odpowiedź:} $\langle E \rangle =  \hbar \omega/2 \cdot  \coth(\hbar \omega / (2k_B T))$. 

\zaddom
Znajdź fluktuacje energii (odchylenie standardowe), tzn,
$\sigma_E^2= \left<E^2\right> - \left<E\right>^2 $ kwantowego oscylatora i
harmonicznego będącego w równowadze z termostatem o temperaturze $T$.
Jaka będzie fluktuacja średniej energii $N$ takich oscylatorów?

\emph{Odpowiedź:} $\sigma_E^2 =  (\hbar \omega/2)^2 \cdot [\sinh(\hbar \omega / (2k_B T))]^{-2}$.

%\zaddom
%W mleku kuleczki tłuszczu o gęstości $\rho_t=0.9\,{\rm g/cm^3}$,
%zawieszone są w wodzie o gęstości $\rho_w=1.0\,{\rm g/cm^3}$.
%Objętość kuleczki tłuszczu jest równa $10\,{\rm \mu m}^3$.
%Stała Boltzmana wynosi $1.38\cdot 10^{-23}\,$J/K.
%\begin{enumerate}[a)]
%\item Jak zależy energia potencjalna kulki tłuszczu od głębokości $H$
%liczonej od powierzchni cieczy?
%\item Jak zależy od $H$ prawdopodobieństwo znalezienia kulki tłuszczu
%w warunkach równowagi termodynamicznej w temperaturze 300\,K?
%\end{enumerate}
%Przyjmij, że głębokość naczynia jest znana.

%\emph{Odpowiedź:}
%a) $
%E(H) =  (\rho_w - \rho_t)V g H.
%$
%b) $p(H)= \alpha  e^{-\alpha H},$
%gdzie 
%$\alpha = 2.4\, \mu m^{-1}$.



\zaddom
{\em Atmosfera adiabatyczna.} Zakładając, że powietrze jest w równowadze hydrostatycznej
oraz że jego temperatura zmienia się z ciśnieniem zgodnie ze wzorem na adiabatę dla gazu
doskonałego, pokazać, że pionowy gradient temperatury jest wtedy stały.
Policzyć ten gradient dla masy molowej powietrza $\mu=29\,$g/mol oraz
$\kappa = C_P/C_V = 7/5$.

\emph{Odpowiedź:}
Gradient temperatury wynosi $\nabla T  = -\frac{\mu g (\kappa-1)}{R\kappa } = 10^{-2}\, {\rm K}/{\rm m}. $ 
%Oznacza to zmianę o $1^\circ {\rm C}$ przy zmianie wysokości o $100\,{\rm m}$.

%%%%%%%%%%%%%%%%%%%%%%%%%
\end{document}
%%%%%%%%%%%%%%%%%%%%%%%%%


%\zadanie
%{\em Przypadek zmiennej gęstości i jednorodnego pola grawitacyjnego.} \\
%W dwóch pionowych rurach o długości 1000\,m, zamkniętych od góry i utrzymywanych w temperaturze
%$0^\circ$C, znajdują się wodór i ksenon. Na dole rur panuje ciśnienie 1\,atm = 1.013\,bar.
%\begin{enumerate}[(a)]
%\item Jakie ciśnienia panują na górze rur?
%\item Przekrój poprzeczny rur $S = 10\,{\rm cm^2}$.
%Jaka wypadkowa siła działa na górne denka rur (zwrot i~wartość)?
%\end{enumerate}
%Gęstości gazów w warunkach normalnych wynoszą: wodór 0.0899\,kg/m$^3$, ksenon 5.89\,kg/m$^3$,
%powietrze 1.293\,kg/m$^3$.
%Przyjąć, że wszystkie gazy można traktować jak gazy doskonałe.

\zaddom
Policzyć ciśnienie wywierane przez $n$ moli gazu doskonałego
złożonego z cząstek o masie $m$ i znajdującego się w objętości $V$,
jeśli rozkład wartości prędkości jest maxwellowski i każdy kierunek prędkości
jest jednakowo prawdopodobny.

\emph{Odpowiedź:} %Rozkład prawdopodobieństwa ciśnienia będzie zadany przez:
%$P(p) = \sqrt{\frac{V}{2 \pi n k T}} \frac{1}{\sqrt{p}} e^{-\frac{V}{2 n k T} p}$,
Średnie ciśnienie $\langle p \rangle = \frac{n k T}{V}$.

\zaddom
Znaleźć rozkład ciśnienia gazu wokół planety o masie $M$ i promieniu $R_0$, przy założeniu,
że gaz ma masę molową $\mu$, a jego temperatura jest wszędzie jednakowa i równa $T_0$,
zaś na powierzchni planety ciśnienie gazu wynosi $p_0$. Podać wzór na stosunek ciśnienia
w granicy nieskończonej odległości od planety, $p_\infty$ do ciśnienia $p_0$.
Wykonać obliczenia dla powietrza, $\mu=29\,$g/mol o temperaturze $T_0 = 273\,$K
w 2 przypadkach:
\begin{enumerate}
\item planety wielkości Ziemi i o masie Ziemi ($R_Z = 6370\,$km, $M_Z = 5.98\cdot10^{24}\,$kg),
\item planety o masie 10 razy mniejszej niż masa Ziemi, ale o takiej samej gęstości.
\end{enumerate}
Przedyskutować otrzymany wynik z punktu widzenia możliwości utrzymywania atmosfery przez planety.
Stała grawitacji $G = 6.67 \cdot 10^{-11}\,{\rm N m^2 kg^{-2}}$,
stała gazowa $R = 8.31\,$J/K.

\emph{Odpowiedź:}
$p(h) = p_0 e^{-\frac{G M \mu}{R T_0} \frac{h}{R_0(R_0+h)}}$, gdzie
$h$ jest odległością od powierzchni planety, 
$p(\infty)/p_0 = e^{-\frac{G M \mu}{R T_0 R_0}}$.  Dla Ziemi:
$p(\infty)/p_0 = e^{-747}$ a dla mniejszej planety: $p(\infty)/p_0 = e^{-161} $. 

\zaddom
Oszacuj temperaturę i ciśnienie w centrum Słońca, przyjmując następujące założenia:
\begin{enumerate}
\item Słońce jest sferycznie symetryczną kulą gazu o promieniu $R_\odot$,
pozostającą w równowadze\linebreak hydrostatycznej
(tzn. w każdym punkcie ciśnienie gazu równoważy ciężar wyższych
jego warstw).
\item Gęstość materii słonecznej zmienia się z odległością od centrum według wzoru:
$\rho(r) = \rho_0 \left( 1 - \frac{r}{R_\odot}\right )$.
\item Materia słoneczna jest gazem doskonałym o masie molowej $\mu = 1.1\,$g/mol.
\item Masa Słońca $M_\odot = 2\cdot 10^{30}\,$kg,
promień Słońca $R_\odot = 7\cdot10^8\,$m.
\end{enumerate}

\emph{Odpowiedź:}
$p_0 = \frac{5}{4 \pi} \frac{G M_\odot}{R_\odot^4} = 4.4 \cdot 10^{14}\, {\rm Pa}$, $T_0 = \frac{p_0 \mu}{R \rho_0}  = 1.05 \cdot 10^7\, {\rm K}$,
gdzie $\rho_0 = \frac{3 M_\odot}{\pi R_\odot^3} = 5.6 \cdot 10^3 {\rm kg}/{\rm m}^3$ jest gęstością gazu wewnątrz Słońca.


% An example of figure placement:
%\begin{wrapfigure}[13]{r}{0.4\linewidth}\vspace{3mm}
%\resizebox{\linewidth}{!}{\includegraphics{NAZWA.png}}
%\end{wrapfigure}
%\zaddom

\end{document}
