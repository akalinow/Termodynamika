\documentclass[11pt,a4paper]{article}

\usepackage[polish]{babel}
\usepackage[utf8]{inputenc}
\usepackage{polski}
\usepackage[T1]{fontenc}
\usepackage{indentfirst}
\usepackage{wrapfig}    % for wrapping figures, tables

\frenchspacing

%\usepackage{amsmath}
\usepackage{physics}
%\usepackage{bm}
\usepackage{gensymb}
%\usepackage{hepnames}
\usepackage{epsfig}
\usepackage{graphics}
\usepackage[shortlabels]{enumitem}
%\usepackage{xspace}
%\xspaceaddexceptions{[]\{\}}

%
%
%fixpagesize
\pagestyle{empty}
\addtolength{\textwidth}{6cm}
\addtolength{\textheight}{4cm}
\addtolength{\evensidemargin}{-3cm}
\addtolength{\oddsidemargin}{-3cm}
\addtolength{\topmargin}{-2cm}
\parindent=0cm


%
%
%small distance in list/item/enum for enumitem package
\setlist[itemize,enumerate]{topsep=0em}
\setlist{noitemsep}

%print zadanie #
\newcounter{zadanie}\newcommand{\zadanie}[1][]{\addtocounter{zadanie}{1} ~\\  {\bf \emph{Zadanie \arabic{zadanie} #1 }} \\}
\newcounter{zaddom}\newcommand{\zaddom}[1][]{\addtocounter{zaddom}{1} ~\\  {\bf \emph{Zadanie domowe \arabic{zaddom} #1 }} \\}
%\renewcommand{\zadanie}[1][]{\pagebreak  ~\\  {\bf \emph{Zadanie }} \\} \addtolength{\topmargin}{-2cm}

\newcommand{\dbar}{{\mkern3mu\mathchar'26\mkern-12mu d}}
\renewcommand{\t}[1]{\textrm{#1}}

%%%%%%%%%%%%%%%%%%%%%%%%%%%%%%%%%%%%%%%%%%%%%%%%%%%%%%
\begin{document}           % End of preamble and beginning of text.

\begin{centering}
\bf{\Large{Termodynamika z elementami fizyki statystycznej}}\\
Tydzień 5  (25 marca 2024)\\[3mm]
przemiany gazowe, praca \\
\end{centering}
\vspace{5mm}

\zadanie
W letni dzień zawieszono na grubej stalowej linie o długości $10\,$m ciężar o masie $20\,$t.
W nocy temperatura spadła o $10^\circ$C. O ile uniósł się ciężar?
Jaka praca została przy tym wykonana? Jaka siła wykonała pracę?
Współczynnik rozszerzalności liniowej stali wynosi $\alpha = 1.2\cdot 10^{-5}\,{\rm K}^{-1}$.
\vskip 10pt
\textbf{Uwaga: }lina wisi pionowo, pomijamy jej masę.

%\textbf{Odpowiedź:} $W=235 \t{J}$

\zadanie
Znaleźć pracę wykonaną przy izotermicznym sprężaniu 1 dm$^3$
gazu doskonałego i bloku miedzianego o tej samej objętości od
ciśnienia $p_1=1\,$atm do $p_2=5\,$atm. Moduł Younga miedzi wynosi $130\,$GPa.
\vskip 10pt
\textbf{Uwaga:} sprężanie bloku miedzianego następuje tylko wzdłuż jednego kierunku.

%\textbf{Odpowiedź:} $W_{\t{gazu}}=161\,\t{J}$, $W_{\t{Cu}} = 9.2 \cdot 10^{-4}\,\t{J}$

\zadanie
$n$ moli gazu doskonałego poddano dwóm przemianom ze stanu początkowego
opisanego parametrami $T_1, V_1$ do stanu końcowego o objętości $V_2$
$(V_2>V_1)$ w ten
sposób, że:\begin{enumerate}
\item $p(V) = \gamma - \alpha (V-V_1)$,
\item $p(V) = \gamma - \beta (V-V_1)^2$.
\end{enumerate}
Współczynniki $\alpha$, $\beta$ i $\gamma$ dobrano tak,
aby końcowe ciśnienie $p_2$ w obu przemianach było jednakowe.
Oblicz: zależnosć $T(V)$, temperaturę końcową $T_2$
oraz pracę wykonaną przez siły zewnętrzne w obu przemianach.
Kiedy $T_2>T_1$?

\vskip 10pt
%\textbf{Odpowiedź:} w przypadku 1): $W =\frac{1}{2}(V_1-V_2)(p_1+p_2)$, w przypadku 2): $W =\frac{1}{3}(V_1-V_2)(2 p_1+p_2)$. Temperatury są w obu przypadkach takie same.


\zadanie
Jeden mol gazu doskonałego przeszedł ze stanu opisanego parametrami
$p_0, V_0$ do stanu o objętości $V_1 = 2 V_0$.
Przemiana prowadzona była tak, że przez cały czas $p^2 V = const$.
Następnie gaz poddano przemianom izochorycznej i izotermicznej, w ten sposób, że gaz osiągnął
początkowe parametry stanu. \\
Znajdź pracę wykonaną nad gazem w każdej z przemian.


\vskip 10pt
%\textbf{Odpowiedź:} Całkowita praca wykonana nad gazem $W = p_0 V_0 [\ln 2 -2(\sqrt{2}-1)] \approx -0.135 \,p_0V_0$ (to gaz efektywnie wykonał pracę)

\zadanie
Oblicz pracę jaką wykona jeden mol gazu van der Waalsa, o równaniu stanu
$(p+\frac{a}{V^2})(V-b)=RT$
($a=0.138\,\rm{Jm}^3/\rm{mol}^2$, $b=31.8\,\rm{cm^3/mol}$)
przy izotermicznym i odwracalnym rozprężaniu od objętości $V_0=22.4\,\rm{dm}^3$
do objętości $2 V_0$ przy temperaturze $T=0\degree$C.
Czy gaz doskonały rozprężając się w takich samych warunkach wykona większą pracę?

\vskip 10pt
%\textbf{Odpowiedź:}  $W = RT \ln(2 + \frac{b}{V_0-b}) - \frac{a}{2 V_0} \approx 1569.1\,\t{J}$ dla porównania gaz doskonały
%$W_{\t{d}} = RT \ln 2 \approx 1570.6\,\t{J}$.


\pagebreak
\zaddom
Stalowa szyna o przekroju poprzecznym $S = 10\,$cm$^2$ i długości $L = 10\,$m zmieniła temperaturę
z $20^\circ$C do $40^\circ$C.
\begin{enumerate}
\item Oszacuj o ile wzrosła długość szyny
\item Jakie siły powinny działać na końce szyny, aby przywrócić jej początkową długość? \\
      Czy wartość tych sił zależy od długości szyny?
\end{enumerate}
Przyjmij, że dla stali: współczynnik liniowej rozszerzalności cieplnej wynosi $\alpha = 1.2\cdot 10^{-5}\,$K$^{-1}$,
zaś moduł Younga wynosi $\displaystyle E = \frac{\Delta p}{\Delta L/L} = 200\,$GPa.

\vskip 10pt
\textbf{Odpowiedź:} $\Delta L = L \alpha \Delta T = 2.4 \cdot 10^{-3}\, {\rm m}$, $F = S E \alpha \Delta T = 4.8 \cdot 10^4\, {\rm N}$.

\zaddom
Jeden mol gazu van der Waalsa ($a=0.138\,\rm{Jm}^3/\rm{mol}^2$, $b=31.8\,\rm{cm^3/mol}$)
utrzymywany jest pod stałym ciśnieniem normalnym w pojemniku o ściankach adiabatycznych, zamkniętym tłokiem,
wypełniając początkowo objętość $V_0=22,4\,\rm{dm}^3$.
W wyniku przesunięcia tłoka objętość jego wzrosła dwukrotnie. Podaj temperturę początkową i końcową.  
Jaki byłby wynik tego typu ekspansji w przypadku gazu doskonałego?

\vskip 10pt
\textbf{Odpowiedź:} 
Temperatura początkowa: $T_0 = \frac{V_0-nb}{nR}\left(p_0 + \frac{n^2 a}{V_0^2} \right)$, temperatura końcowa $T_1 = T_0 \left(\frac{V_1 - nb}{V_0 - nb} \right)^{nR/C_v}$. 

\zaddom
Równanie stanu paramagnetyka (np. w polu solenoidu) zapisać można jako:
\[  {\rm M}= C \frac{H}{T}, \]
gdzie $C$ - stała Curie, $H$ - natężenia pola magnetycznego, $T$- temperatura,
$\rm M$ - magnetyzacja, tj. moment magnetyczny $M$ w jednostce objętości (${\rm M}=M/V$).
Praca wykonana nad paramagnetykiem w polu magnetycznym
 (z wyłączeniem energii wzajemnej paramegantyk-solenoid) może być zapisana jako
 $\dbar W = -\mu_0 M dH$.
Znajdź pracę: 
\begin{itemize}
\item w przemianie izotermicznej przy zmianie momentu magnetycznego od $M_1$ do $M_2$
\item w przemianie dla której $H/T={\rm const}$, od $T_1$ do $T_2$. 
\end{itemize}

\vskip 10pt
\textbf{Odpowiedź:}
\begin{itemize}
	\item $W = - \frac{1}{2} \frac{\mu_0 T}{C} \left(M_2^2 - M_1^2 \right)$,
	\item $W = - \frac{\mu_0 M^2}{C} (T_2 - T_1)$.
\end{itemize}

\end{document}
